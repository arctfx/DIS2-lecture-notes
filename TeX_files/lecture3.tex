\documentclass[12pt]{article}
\usepackage[a4paper, margin=3cm]{geometry}
\usepackage[T2A,T1]{fontenc}
\usepackage[utf8]{inputenc}
\usepackage[russian,english]{babel}

\usepackage{amsmath}
\usepackage{amsfonts}
\usepackage{graphicx}
\usepackage{epstopdf}
\usepackage{amssymb}
\usepackage[utf8]{inputenc}
\DeclareUnicodeCharacter{25A9}{\dash}
\usepackage{color}
\definecolor{darkgray}{gray}{0.1}

\newtheorem{proposition}{Твърдение}
\newtheorem{corollary}{Следствие}
\newtheorem{definition}{Def.}
\newtheorem{lemma}{Lemma}
\newtheorem{theorem}{Th.}
\newcommand{\suma}[2]{\overset{#2}{\underset{#1}{\sum}}}
\newcommand{\spc}{\text{ }}

\everymath{\displaystyle}

%Riemann integrals
\def\upint{\mathchoice%
	{\mkern13mu\overline{\vphantom{\intop}\mkern7mu}\mkern-20mu}%
	{\mkern7mu\overline{\vphantom{\intop}\mkern7mu}\mkern-14mu}%
	{\mkern7mu\overline{\vphantom{\intop}\mkern7mu}\mkern-14mu}%
	{\mkern7mu\overline{\vphantom{\intop}\mkern7mu}\mkern-14mu}%
	\int}
\def\lowint{
	\mkern3mu\underline{\vphantom{\intop}\mkern7mu}\mkern-10mu\int}

\begin{document}
	\selectlanguage{russian}
	\color{white}
	\pagecolor{darkgray}
	\title{Записки по ДИС2 - Лекция 3}
	\maketitle
	\textbf{ }
	
	%LEMMA1
	\begin{lemma}
		Нека $f,g: \left[a,\spc b\right] \mapsto \mathbb{R}$ са интегруеми. Тогава $f\circ g$ е интегруема в интервала $\left[a,\spc b\right]$.
	\end{lemma}
	% BEGIN PROOF
	\textbf{Доказателство:} Нека $\tau : a=x_{0}<x_{1}<x_{2}<...<x_{n}=b$ и $\epsilon>0$. Също така $x,\spc y\in\left[x_{i-1},\spc x_{i}\right]$. Разглеждаме разликата между сумите на Дарбу на композицията на $f$ и $g$:
	\[S_{fg}(\tau)-s_{fg}(\tau)=\sum_{i=1}^{n}\omega (fg;\left[x_{i-1},\spc x_{i}\right])(x_{i} - x_{i-1})\].
	........
	$\blacksquare$
	% END PROOF
	
	\begin{corollary}
		\textbf{от Теоремата за средните стойности}: Нека $f: \left[a,\spc b\right] \mapsto \mathbb{R}$ е непрекъсната и $g: \left[a,\spc b\right] \mapsto \mathbb{R}$ е интегруема, като $g$ е неотрицателна в интервала $\left[a,\spc b\right]$. Тогава съществува $\xi\in\left[a,\spc b\right]$, за което $\int_{a}^{b} f(x)\,\mathrm{d}x = f(\xi)\int_{a}^{b} g(x)\,\mathrm{d}x$.
	\end{corollary}
	% BEGIN PROOF
	\textbf{Доказателство:} \textbf{(Iсл.)} Нека ...\\
	\textbf{(Iсл.)}........
	$\blacksquare$
	% END PROOF
	
	
\end{document}