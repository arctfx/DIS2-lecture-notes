\documentclass[12pt]{article}
\usepackage[a4paper, margin=2.5cm]{geometry}
\usepackage[T2A,T1]{fontenc}
\usepackage[utf8]{inputenc}
\usepackage[russian,english]{babel}

\usepackage{amsmath}
\usepackage{amsfonts}
\usepackage{graphicx}
\usepackage{epstopdf}
\usepackage{amssymb}
\usepackage[utf8]{inputenc}
\DeclareUnicodeCharacter{25A9}{\dash}
\usepackage{color}
\definecolor{darkgray}{gray}{0.1}

\newtheorem{proposition}{Твърдение}
\newtheorem{definition}{Def.}
\newtheorem{lemma}{Lemma}
\newtheorem{theorem}{Th.}
\newcommand{\suma}[2]{\overset{#2}{\underset{#1}{\sum}}}
\newcommand{\spc}{\text{ }}

\everymath{\displaystyle}

%Riemann integrals
\def\upint{\mathchoice%
	{\mkern13mu\overline{\vphantom{\intop}\mkern7mu}\mkern-20mu}%
	{\mkern7mu\overline{\vphantom{\intop}\mkern7mu}\mkern-14mu}%
	{\mkern7mu\overline{\vphantom{\intop}\mkern7mu}\mkern-14mu}%
	{\mkern7mu\overline{\vphantom{\intop}\mkern7mu}\mkern-14mu}%
	\int}
\def\lowint{
	\mkern3mu\underline{\vphantom{\intop}\mkern7mu}\mkern-10mu\int}

\begin{document}
	\selectlanguage{russian}
	\color{white}
	\pagecolor{darkgray}
	\title{Записки по ДИС2 - Лекция 4}
	\date{16.03.2023}
	\maketitle
	\begin{center}
		\Large
		\textbf{Криволинеен трапец. Полярни координати. Обем на ротационно тяло. Лице на околна повърхнина на ротационно тяло}
	\end{center}

	%###########################################################
	\begin{definition} \textbf{\textit{Криволинеен трапец}}
		Нека $D = \{(x,y)\in \mathbb{R}: f(x)\leq y\leq g(x), a \leq x \leq b\}$
	\end{definition}
	..................
	
\end{document}
