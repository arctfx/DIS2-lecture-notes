\documentclass[12pt]{article}
\usepackage[a4paper, margin=2.5cm]{geometry}
\usepackage[T2A,T1]{fontenc}
\usepackage[utf8]{inputenc}
\usepackage[russian,english]{babel}
\usepackage{wrapfig}
\usepackage{amsmath}
\usepackage[customcolors]{hf-tikz}
\usepackage{tikz}
\usetikzlibrary{patterns}
\usetikzlibrary {patterns.meta}
\usepackage{tcolorbox}
\usepackage{amsfonts}
\usepackage{mathtools}
\usepackage{graphicx}
\usepackage{epstopdf}
\usepackage{amssymb}
\usepackage{cancel}
\usepackage{hf-tikz}
\usepackage{pgfplots}
\usepackage[utf8]{inputenc}
\DeclareUnicodeCharacter{25A9}{\dash}
\usepackage{color}
\definecolor{darkgray}{gray}{0.1}

%TikZ hf-tikz
\hfsetfillcolor{teal!50!black}
\hfsetbordercolor{teal!80}

%halfbox
\newtcbox{\rbbox}[1][]{on line, sharp corners, colframe=white,coltext=white, colback=darkgray, size=small, toprule=0pt, leftrule=0pt, #1}
\newcommand{\halfbox}[1]{\rbbox{#1}\quad}


\everymath{\displaystyle}

%Riemann integrals
\def\upint{\mathchoice%
	{\mkern13mu\overline{\vphantom{\intop}\mkern7mu}\mkern-20mu}%
	{\mkern7mu\overline{\vphantom{\intop}\mkern7mu}\mkern-14mu}%
	{\mkern7mu\overline{\vphantom{\intop}\mkern7mu}\mkern-14mu}%
	{\mkern7mu\overline{\vphantom{\intop}\mkern7mu}\mkern-14mu}%
	\int}
\def\lowint{
	\mkern3mu\underline{\vphantom{\intop}\mkern7mu}\mkern-10mu\int}
\newcommand{\RNum}[1]{\uppercase\expandafter{\romannumeral #1\relax}}

%NEWTHEOREMS
\newtheorem{proposition}{Твърдение}
\newtheorem{definition}{Def.}
\newtheorem{lemma}{Lemma}
\newtheorem{theorem}{Th.}
\newcommand{\suma}[2]{\overset{#2}{\underset{#1}{\sum}}}
\newcommand{\spc}{\text{ }}

%BEGIN DOCUMENT
\begin{document}
	\selectlanguage{russian}
	\color{white}
	\pagecolor{darkgray}
	\title{Записки по ДИС2 - Лекция 11 Онлайн}
	\date{04.05.2023}
	\maketitle
	\begin{center}
		\Large
		\textbf{Топология на $\mathbb{R}$.}
	\end{center}
	
	%###########################################################
	\begin{math}
		\mathbb{R}^n = \{x=(x_1, ..., x_n)\spc:\spc x_i\in\mathbb{R},\spc i\in\{1,...,n\}\}\quad\quad
		x = (x_1, ..., x_n)\quad\quad
		\mathcal{O} = (0, ..., 0)\\
		\text{\underline{Евклидова норма}: }\|x\| := \sqrt{\sum_{i=1}^{n}x_i^2}\quad\rightarrow\quad \|x\|=\sqrt{\langle x, x \rangle}\\
		B_r(x) = \{y\in\mathbb{R}^n : \|x-y\| < r\}\quad\quad
		\overline{B_r}(x) = \{y\in\mathbb{R}^n : \|x-y\| \leq r\}\\
		\|x\|_p = \left(\sum_{i=1}^{n}|x_i|^p \right) ^{\frac{1}{p}},\quad p\geq 1\quad\quad\quad
		\|x\|_{\infty} = \max\{|x_1|,...,|x_n|\}
	\end{math}
\end{document}