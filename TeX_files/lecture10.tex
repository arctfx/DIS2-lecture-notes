\documentclass[12pt]{article}
\usepackage[a4paper, margin=2.5cm]{geometry}
\usepackage[T2A,T1]{fontenc}
\usepackage[utf8]{inputenc}
\usepackage[russian,english]{babel}
\usepackage{wrapfig}
\usepackage{amsmath}
\usepackage[customcolors]{hf-tikz}
\usepackage{tikz}
\usetikzlibrary{patterns}
\usetikzlibrary {patterns.meta}
\usepackage{tcolorbox}
\usepackage{amsfonts}
\usepackage{mathtools}
\usepackage{graphicx}
\usepackage{epstopdf}
\usepackage{amssymb}
\usepackage{cancel}
\usepackage{hf-tikz}
\usepackage{pgfplots}
\usepackage[utf8]{inputenc}
\DeclareUnicodeCharacter{25A9}{\dash}
\usepackage{color}
\definecolor{darkgray}{gray}{0.1}

%TikZ hf-tikz
\hfsetfillcolor{teal!50!black}
\hfsetbordercolor{teal!80}

%halfbox
\newtcbox{\rbbox}[1][]{on line, sharp corners, colframe=white,coltext=white, colback=darkgray, size=small, toprule=0pt, leftrule=0pt, #1}
\newcommand{\halfbox}[1]{\rbbox{#1}\quad}


\everymath{\displaystyle}

%Riemann integrals
\def\upint{\mathchoice%
	{\mkern13mu\overline{\vphantom{\intop}\mkern7mu}\mkern-20mu}%
	{\mkern7mu\overline{\vphantom{\intop}\mkern7mu}\mkern-14mu}%
	{\mkern7mu\overline{\vphantom{\intop}\mkern7mu}\mkern-14mu}%
	{\mkern7mu\overline{\vphantom{\intop}\mkern7mu}\mkern-14mu}%
	\int}
\def\lowint{
	\mkern3mu\underline{\vphantom{\intop}\mkern7mu}\mkern-10mu\int}
\newcommand{\RNum}[1]{\uppercase\expandafter{\romannumeral #1\relax}}

%NEWTHEOREMS
\newtheorem{proposition}{Твърдение}
\newtheorem{definition}{Def.}
\newtheorem{lemma}{Lemma}
\newtheorem{theorem}{Th.}
\newcommand{\suma}[2]{\overset{#2}{\underset{#1}{\sum}}}
\newcommand{\spc}{\text{ }}

%BEGIN DOCUMENT
\begin{document}
	\selectlanguage{russian}
	\color{white}
	\pagecolor{darkgray}
	\title{Записки по ДИС2 - Лекция 10}
	\date{27.04.2023}
	\maketitle
	\begin{center}
		\Large 
		\textbf{Функции на няколко променливи.}
	\end{center}
	
	%###########################################################
	\begin{align*}\mathbb{R}^{n} = &\{x=(x_1,x_2,...,x_n):x_i\in\mathbb{R} \spc\forall i\in\{1,...,n\}\}\\
	&x\in\mathbb{R}\\
	\spc\\
	&x=(x_1,x_2,...,x_n)\\
	&y=(y_1,y_2,...,y_n)\\
	&x+y=(x_1+y_1,x_2+y_2,...,x_n+y_n)\\
	&x\in\mathbb{R}\\
	&\lambda x =(\lambda x_1, \lambda x_2, ..., \lambda x_n)\\
	&\mathcal{O} = (0,0,...,0)\\
	\end{align*}

	\begin{definition}
		\underline{Норма}\\
		$\|.\|:\mathbb{R}^n \rightarrow [a,+\infty)$ е норма в $\mathbb{R}^n$, ако са изпълнени следните условия:\\
		1) $\|x\|=0 \Leftrightarrow x=\mathcal{O}$\\
		2) $\|\lambda x\| = |\lambda|.\|x\|\quad\forall\lambda\in\mathbb{R}^n\spc\forall x \in\mathbb{R}^n$\\
		3) $\|x+y\|\leq \|x\|+\|y\|\quad\forall x,y \in\mathbb{R}^n$ (неравенство на триъгълника)
	\end{definition}
	$\|x-y\|=\|(x-z)+(z-y)\|\leq\|x-z\|+\|z-y\|$
	\begin{definition}
		\underline{Евклидова норма}\\
		$\|x\|=\sqrt{\sum_{i=1}^{n}x_i^2}$ - евклидова норма в $\mathbb{R^n}$\\
		$d(x,y) = \|y-x\|$
	\end{definition}
	\begin{definition}
		\underline{Скаларно произведение}\\
		$\langle x,y \rangle = \sum_{i=1}^{n}x_i y_i\quad \rightarrow \|x\| = \sqrt{\langle x, y \rangle}$\\
		$\spc$\\
		Свойство на скаларното произведение: \textbf{Неравенство на Коши-Буняковски-Шварц}\\
		\[|\langle x, y \rangle| \leq \|x\|.\|y\|\quad \forall x,y \in \mathbb{R}^n\]
		$\spc$\\
		Алтернативен запис на неравеството е:\\
		$\left|\sum_{i=1}^{n}x_iy_i\right|\leq \sqrt{\sum_{i=1}^{n}x_i^2}.\sqrt{\sum_{i=1}^{n}y_i^2}$\\
		
	\end{definition}
	$\lambda\in\mathbb{R}\quad 0\leq\|x+\lambda y\|^2 = \langle x+\lambda y, x+\lambda y \rangle = $\\
	$\|x\|^2 + 2\lambda \langle x,y \rangle +\lambda^2.\|y\|^2$\\
	$(2\langle x,y \rangle)^2 - 4\|x |^2.\|y\|^2 \leq 0$\\
	$(\langle x,y \rangle)^2 \leq \|x\|^2.\|y\|^2$\\
	$\spc$\\
	$\|x+y\|^2 = \|x\|^2 + 2\langle \rangle nnmk$
\end{document}