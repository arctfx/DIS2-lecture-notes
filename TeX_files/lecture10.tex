\documentclass[12pt]{article}
\usepackage[a4paper, margin=2.5cm]{geometry}
\usepackage[T2A,T1]{fontenc}
\usepackage[utf8]{inputenc}
\usepackage[russian,english]{babel}
\usepackage{wrapfig}
\usepackage{amsmath}
\usepackage[customcolors]{hf-tikz}
\usepackage{tikz}
\usetikzlibrary{patterns}
\usetikzlibrary {patterns.meta}
\usepackage{tcolorbox}
\usepackage{amsfonts}
\usepackage{mathtools}
\usepackage{graphicx}
\usepackage{epstopdf}
\usepackage{amssymb}
\usepackage{cancel}
\usepackage{hf-tikz}
\usepackage{pgfplots}
\usepackage[utf8]{inputenc}
\DeclareUnicodeCharacter{25A9}{\dash}
\usepackage{color}
\definecolor{darkgray}{gray}{0.1}

%TikZ hf-tikz
\hfsetfillcolor{teal!50!black}
\hfsetbordercolor{teal!80}

%halfbox
\newtcbox{\rbbox}[1][]{on line, sharp corners, colframe=white,coltext=white, colback=darkgray, size=small, toprule=0pt, leftrule=0pt, #1}
\newcommand{\halfbox}[1]{\rbbox{#1}\quad}


\everymath{\displaystyle}

%Riemann integrals
\def\upint{\mathchoice%
	{\mkern13mu\overline{\vphantom{\intop}\mkern7mu}\mkern-20mu}%
	{\mkern7mu\overline{\vphantom{\intop}\mkern7mu}\mkern-14mu}%
	{\mkern7mu\overline{\vphantom{\intop}\mkern7mu}\mkern-14mu}%
	{\mkern7mu\overline{\vphantom{\intop}\mkern7mu}\mkern-14mu}%
	\int}
\def\lowint{
	\mkern3mu\underline{\vphantom{\intop}\mkern7mu}\mkern-10mu\int}
\newcommand{\RNum}[1]{\uppercase\expandafter{\romannumeral #1\relax}}

%NEWTHEOREMS
\newtheorem{proposition}{Твърдение}
\newtheorem{definition}{Def.}
\newtheorem{lemma}{Lemma}
\newtheorem{theorem}{Th.}
\newcommand{\suma}[2]{\overset{#2}{\underset{#1}{\sum}}}
\newcommand{\spc}{\text{ }}

%BEGIN DOCUMENT
\begin{document}
	\selectlanguage{russian}
	\color{white}
	\pagecolor{darkgray}
	\title{Записки по ДИС2 - Лекция 10}
	\date{27.04.2023}
	\maketitle
	\begin{center}
		\Large 
		\textbf{Функции на няколко променливи.}
	\end{center}
	
	%###########################################################
	\begin{align*}\mathbb{R}^{n} = &\{x=(x_1,x_2,...,x_n):x_i\in\mathbb{R} \spc\forall i\in\{1,...,n\}\}\\
	&x\in\mathbb{R}\\
	\spc\\
	&x=(x_1,x_2,...,x_n)\\
	&y=(y_1,y_2,...,y_n)\\
	&x+y=(x_1+y_1,x_2+y_2,...,x_n+y_n)\\
	&x\in\mathbb{R}\\
	&\lambda x =(\lambda x_1, \lambda x_2, ..., \lambda x_n)\\
	&\mathcal{O} = (0,0,...,0)\\
	\end{align*}

	\begin{definition}
		\underline{Норма}\\
		$\|.\|:\mathbb{R}^n \rightarrow [a,+\infty)$ е норма в $\mathbb{R}^n$, ако са изпълнени следните условия:\\
		1) $\|x\|=0 \Leftrightarrow x=\mathcal{O}$\\
		2) $\|\lambda x\| = |\lambda|.\|x\|\quad\forall\lambda\in\mathbb{R}^n\spc\forall x \in\mathbb{R}^n$\\
		3) $\|x+y\|\leq \|x\|+\|y\|\quad\forall x,y \in\mathbb{R}^n$ (неравенство на триъгълника)
	\end{definition}
	$\|x-y\|=\|(x-z)+(z-y)\|\leq\|x-z\|+\|z-y\|$
	\begin{definition}
		\underline{Евклидова норма}\\
		$\|x\|=\sqrt{\sum_{i=1}^{n}x_i^2}$ - евклидова норма в $\mathbb{R}^n $\\
		$d(x,y) = \|y-x\|$
	\end{definition}
	\begin{definition}
		\underline{Скаларно произведение}\\
		$\langle x,y \rangle = \sum_{i=1}^{n}x_i y_i\quad \rightarrow \|x\| = \sqrt{\langle x, y \rangle}$\\
		$\spc$\\
		Свойство на скаларното произведение: \textbf{Неравенство на Коши-Буняковски-Шварц}\\
		\[|\langle x, y \rangle| \leq \|x\|.\|y\|\quad \forall x,y \in \mathbb{R}^n\]
		$\spc$\\
		Алтернативен запис на неравеството е:\\
		$\left|\sum_{i=1}^{n}x_iy_i\right|\leq \sqrt{\sum_{i=1}^{n}x_i^2}.\sqrt{\sum_{i=1}^{n}y_i^2}$\\
		
	\end{definition}
	$\lambda\in\mathbb{R}\quad 0\leq\|x+\lambda y\|^2 = \langle x+\lambda y, x+\lambda y \rangle = $\\
	$\|x\|^2 + 2\lambda \langle x,y \rangle +\lambda^2.\|y\|^2$\\
	$(2\langle x,y \rangle)^2 - 4\|x |^2.\|y\|^2 \leq 0$\\
	$(\langle x,y \rangle)^2 \leq \|x\|^2.\|y\|^2$\\
	$\spc$\\
	$\|x+y\|^2 = \|x\|^2 + 2\langle \rangle nnmk$
	\\
	
	%###############################################
	\begin{center}
		\textbf{Функция на n променливи}
	\end{center}
	\begin{math}
		f:D\rightarrow \mathbb{R}, \quad D\subset\mathbb{R} \quad
		f(x)\equiv f(x_1,x_2,...,x_n)\\
		\\
		f:D\rightarrow \mathbb{R}^k, \quad D\subset\mathbb{R}^n \spc(n,k \in \mathbb{R})\quad
		f(x)=f(x_1,...,x_n)=\begin{pmatrix}
			f_1(x_1,...,x_n)\\
			f_2(x_1,...,x_n)\\
			... \spc... \spc... \spc...\\
			f_k(x_1,...,x_k)
		\end{pmatrix}\quad x\in D\\
		\\
		f_1,f_2,...,f_k \text{ - координатни функции на изображението }f
	\end{math}
	Toчка на сгъстяване на f:
	\[\forall \spc U\text{ - околност на }x_0 : (U\cap D)\{x_0\}\neq\emptyset \text{ или } \{x_m\}_{m=1} ^\infty\subset D\ \{x_0\}, x_m\underset{m\to\infty}{\longrightarrow}x_0\]
	\[\lim\limits_{x\to x_0}f(x)=l,\quad l\in\mathbb{R}^k\]
	\begin{flalign*}
		\text{Коши }&\spc\forall\epsilon>0 \spc\exists\delta>0 \spc\forall x \in D,\spc x\neq x_0, \|x-x_0\|<\delta:\|f(x)-l\|<\epsilon\\
		\text{Хайне }&\spc\forall\{x_m\}_{m=1}^{\infty}\subset D \textbackslash \{x_0\},\spc x_m\underset{m\to\infty}{\longrightarrow}x_0 : f(x_m)\underset{m\to\infty}{\longrightarrow}l
	\end{flalign*}
	\\
	\\
	$f:D\rightarrow \mathbb{R}^k, \quad D\subset\mathbb{R}^n\quad$
	$f \text{ е непрекъсната в } x_0, \text{ ако: }\\$
	\begin{flalign*}&\forall\epsilon>0 \spc\exists\delta>0 \spc\forall x\in D, \spc\|x-x_0\|<\delta:\|f(x)-f(x_0)\|<\epsilon \Leftrightarrow\\
		&\forall\{x_m\}_{m=1}^\infty \subset D,\spc x_m\underset{m\to\infty}{\rightarrow}x_0 : f(x_m)\underset{m\to\infty}{\rightarrow}f(x_0)
	\end{flalign*}
	\halfbox{Пример} $f(x)= \begin{cases}
		\quad\frac{x_1x_2}{x_1^2+x_2^2},\quad &x=(x_1,x_2)\neq\overset{\rightarrow}{\mathcal{O}}=(0,0)\\
		\quad0,&\text{ако }x=(0,0)
	\end{cases}$\\
	$f:\mathbb{R}^2\rightarrow\mathbb{R}$\\
	$x_2 = kx_1,\spc k\in\mathbb{R}$\\
	$f(t,kt)=\frac{t.kt}{t^2+(kt)^2}=\frac{k}{1+k_2}\quad t\neq0$\\
	\newpage
	\begin{center}
		\textbf{Основни теореми за непрекъснати функции (и изображения)}
	\end{center}
	\begin{theorem}
		\underline{Вайерщрас}: Непрекъснат образ на компакт е компакт.\\
		$\begin{cases}
			f: K\rightarrow\mathbb{R}^k,\spc K\subset\mathbb{R}^n, K-\text{компакт}\\
			f \text{ е непрекъснато }\Rightarrow f(K) = \{f(x):x\in K\} \text{ е компактно подмножество на }\mathbb{R}^k
		\end{cases}$\\
		\\
		\underline{Коментар}: $f:[a,b]\rightarrow \mathbb{R}$ - непрекъсната $\Rightarrow f([a,b])$ - компакт в $\mathbb{R}^1\rightarrow f([a,b])$ - ограничено в $\mathbb{R}$, т.е. $f$ е ограничена\\
		$f([a,b])$ - затворено в $\mathbb{R}^1$(значи $\sup f([a,b])$ и $\inf f([a,b])$ са от $f([a,b])$)	
	\end{theorem}
	%BEGIN PROOF
	\underline{Доказателство:}
	%END PROOF
\end{document}