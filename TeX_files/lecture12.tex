\documentclass[12pt]{article}
\usepackage[a4paper, margin=2.5cm]{geometry}
\usepackage[T2A,T1]{fontenc}
\usepackage[utf8]{inputenc}
\usepackage[russian,english]{babel}

\usepackage{wrapfig}
\usepackage{amsmath}
\usepackage[customcolors]{hf-tikz}
\usepackage{tikz}
\usetikzlibrary{patterns}
\usetikzlibrary {patterns.meta}
\usepackage{tcolorbox}
\usepackage{amsfonts}
\usepackage{mathtools}
\usepackage{graphicx}
\usepackage{epstopdf}
\usepackage{amssymb}
\usepackage{cancel}
\usepackage{hf-tikz}
\usepackage{pgfplots}
\usepackage[utf8]{inputenc}\usepackage{wrapfig}
\usepackage{amsmath}
\usepackage[customcolors]{hf-tikz}
\usepackage{tikz}
\usetikzlibrary{patterns}
\usetikzlibrary {patterns.meta}
\usepackage{tcolorbox}
\usepackage{amsfonts}
\usepackage{mathtools}
\usepackage{graphicx}
\usepackage{epstopdf}
\usepackage{amssymb}
\usepackage{cancel}
\usepackage{hf-tikz}
\usepackage{pgfplots}
\usetikzlibrary{shadings}
\usepackage[utf8]{inputenc}
\DeclareUnicodeCharacter{25A9}{\dash}
\usepackage{color}
\definecolor{darkgray}{gray}{0.1}

\everymath{\displaystyle}

%Riemann integrals
\def\upint{\mathchoice%
	{\mkern13mu\overline{\vphantom{\intop}\mkern7mu}\mkern-20mu}%
	{\mkern7mu\overline{\vphantom{\intop}\mkern7mu}\mkern-14mu}%
	{\mkern7mu\overline{\vphantom{\intop}\mkern7mu}\mkern-14mu}%
	{\mkern7mu\overline{\vphantom{\intop}\mkern7mu}\mkern-14mu}%
	\int}
\def\lowint{
	\mkern3mu\underline{\vphantom{\intop}\mkern7mu}\mkern-10mu\int}
\newcommand{\RNum}[1]{\uppercase\expandafter{\romannumeral #1\relax}}
%Halfbox
\newcommand{\halfbox}[1]{\underline{\textbf{#1}:}\textbf{\large{| }}}


%NEWTHEOREMS
\newtheorem{proposition}{Твърдение}
\newtheorem{definition}{Def.}
\newtheorem{lemma}{Lemma}
\newtheorem{theorem}{Th.}
\newcommand{\suma}[2]{\overset{#2}{\underset{#1}{\sum}}}
\newcommand{\spc}{\text{ }}

%BEGIN DOCUMENT
\begin{document}
	\selectlanguage{russian}
	\color{white}
	\pagecolor{darkgray}
	\title{Записки по ДИС2 - Лекция 12}
	\date{11.05.2023}
	\maketitle
	\begin{center}
		\Large
		\textbf{Диференцируемост. Частни производни. Градиент. Производна по направление. Гладки функции}
	\end{center}
	
	%###########################################################
	
	\section*{Геометрична представа}
	Производната е локална линейна апроксимация на функцията.\\
	$f(x) = f(x_0)+f'(x_0)(x-x_0)+R(x;\spc x_0)$\\
	$\frac{R(x;\spc x_0)}{|x-x_0|}\underset{x\to x_0}{\longrightarrow}0$\\
	$\spc$\\
	.....\\
	
	\section*{Диференцируемост}
	\begin{definition}
		$U$ - отворено множество в $\mathbb{R}^n, x_0\in U\quad f:U\rightarrow\mathbb{R}$\\
		Казваме, че $f$ е диференцируема в $x_0$, ако съществува линеен оператор $\partial f(x_0)$ (нарича се диференциал на $f$ в т. $x_0$) такъв, че \\
		$f(x)=f(x_0)+\partial f(x_0)(x-x_0)+o(\|x-x_0\|)$ или\\
		$\lim\limits_{x\to x_0}\frac{f(x)-f(x_0)-\partial f(x_0)(x-x_0)}{\|x-x_0\|}=0$
	\end{definition}
	$f(x_0 +h) = f(x_0) + df(x_0)h + o(\|h\|)\quad$ или 
	$\quad\lim\limits_{h\to\mathcal{O}}\frac{f(x_0 +h) - f(x_0) - df(x_0)(h)}{\|h\|} = 0$\\
	
	\begin{align*}
		&df(x_0) : \mathbb{R}^n \rightarrow \mathbb{R}\\
		&e_1 = (1,0,...,0)\\
		&e_2 = (0,1,...,0)\\
		&	.....\\
		&e_i = (0,..,0,\underset{i\text{-та позиция}}{1},0,..,0)\\
		&h = (h_1,h_2,...,h_n)
	\end{align*}
	$df(x_0)(h)=df(x_0)\left(\sum_{i=1}^{n}h_i e_i\right) = \sum_{i=1}^{n}h_i . df(x_i)(e_i) = $
	$\big\langle \spc (\spc df(x_0)(e_1),\spc df(x_0)(e_2),\spc...,\spc df(x_0)(e_n)\spc),\quad h\spc\big\rangle$
	
	\section*{Частни производни}
	\begin{definition}
		\underline{Частна производна}\\
		Нека $\{x_0 + \lambda e_i : \lambda \in \mathbb{R}\}\text{ - права през }x_0\text{, успоредна на }e_i$. 
		$\frac{\partial f}{\partial x_i}(x_0) := \lim\limits_{\lambda\to0}\frac{f(x_0+\lambda e_i)}{\lambda}$ ще наричаме частна производна на $f$ в т. $x_0$ по $x_i$.
	\end{definition}
	\halfbox{Пример}: Частна производна на $f$ в т. $x_0$ по $x_1$:\\
	$\frac{\partial f}{\partial x_1}(x_0) = \lim\limits_{\lambda to 0}\frac{f(x_1^0+\lambda, x_2^0, ..., x_n^0)-f(x_1^0, ..., x_n^0)}{\lambda}$, където точката $x_0 = (x_1^0, ..., x_n^0)$.
	
	\begin{proposition}
		Ако $f$ e диференцируема в $x_0$, то $\frac{\partial f}{\partial x_i}(x_0)$ съществува за всяко $i\in\{1,...,n\}$. При това $df(x_0)(e_i)=\frac{\partial f}{\partial x_i}(x_0)$.
	\end{proposition}
	%BEGIN PROOF
	\underline{Доказателство}:\\
	\begin{math}
		\lim\limits_{\lambda\to0}\frac{f(x_0+\lambda e_i)-f(x_0)}{\lambda} = \lim\limits_{\lambda\to0}\frac{f(x_0+\lambda e_i)-f(x_0)-\partial f(x_0)(\lambda e_i)+\partial f(x_0)(\lambda e_i)}{\lambda} = \\
		\lim\limits_{\lambda\to 0}\left[\frac{f(x_0+\lambda e_i) - f(x_0 - df(x_0)(\lambda e_i))}{\|\lambda e_i\|}.sgn(\lambda) + df(x_0)(e_i)\right] =
		0 + df(x_0)(e_i)
	\end{math} \begin{flushright}
		$\square$
	\end{flushright}
	%END PROOF
	
	
	\section*{Градиент}
	\begin{definition}
		\underline{Градиент}\\
		$grad f(x_0) := \left( \frac{\partial f}{\partial x_1}(x_0), \frac{\partial f}{\partial x_2}(x_0), ..., \frac{\partial f}{\partial x_n}(x_0)\right)$, \color{gray}//алтернативно ще се отбелязва с $\nabla$
		\color{white}
	\end{definition}
	$\spc$\\
	\[\tikzmarkin[set fill color=red!30!black, set border color=red!30!purple]{z1}(0.5,-0.45)(-0.5,0.7)
	df(x_0)(h) = \langle grad f(x_0), h\rangle
	\tikzmarkend{z1}\]\\

	\begin{proposition} \textbf{Диференцируемите функции са непрекъснати}\\
		$U\subset\mathbb{R}^n$ - отворено множество; нека $f:U\rightarrow\mathbb{R}$ и $f$ е диференцируема в $x_0\in U$. Тогава $f$ е непрекъсната в $x_0$.
	\end{proposition}
	\underline{Доказателство}:\\
	\begin{math}
		\lim\limits_{x\to x_0}f(x) = \lim\limits_{x\to x_0}(f(x)-f(x_0)) + f(x_0) = f(x_0) + \lim\limits_{x\to x_0}\frac{f(x)-f(x_0)-df(x_0)(x-x_0)}{\|x-x_0\|}.\\
		.\|x-x_0\| + \lim\limits_{x\to x_0}df(x_0)(x-x_0) = f(x_0) + 0 + 0
	\end{math}\\
	(Линейните оператори в $\mathbb{R}^n$ са непрекъснати)
	
	\section*{Производна по направление}
	\begin{math}
		U\subset \mathbb{R}^n, U - \text{отворено, } x_0 \in U, f:U\rightarrow\mathbb{R},\spc l\in\mathbb{R}^n \spc \{x_0+\lambda l:\lambda \in \mathbb{R}\}\\
		\{\lambda \in \mathbb{R}: x_0 + \lambda l \in U\} - \text{околност на }0\\
		B_\delta (x_0) \subset U\quad
		x_0 + \lambda l \in B_\delta (x_0) \Leftrightarrow \|\lambda l\|<\delta \Leftrightarrow |\lambda|.\|l\|<\delta\\
		\spc\\
		\frac{\partial f(x_0)}{\partial l} := \lim\limits_{\lambda \to 0}\frac{f(x_0 +\lambda l) - f(x_0)}{\lambda} \quad\longleftarrow \quad
	\end{math} производна на $f$ в т. $x_0$ по направление $l$
	\begin{proposition}
		Ако $f$ е диференцируема в $x_0$, то $f$ има производна в $x_0$ по всяко направление. При това\\
		\[\frac{\partial f}{\partial l}(x_0)=df(x_0)(l)=<\nabla f(x_0),\spc l>\] 
		
		\underline{Забележка}: със символа $\nabla$ ще се отбелязва градиент на функция.
	\end{proposition}
	\begin{math}
		\lim\limits_{\lambda\to 0}\frac{f(x_0+\lambda l) - f(x_0)}{\lambda} = \lim\limits_{\lambda\to 0}\frac{f(x_0 +\lambda l) - f(x_0) - df(x_0)(\lambda l) + df(x_0)(\lambda l)}{\lambda} =\\
		\lim\limits_{\lambda\to 0}\left[\frac{f(x_0 +\lambda l) - f(x_0) -\lambda df(x_0)(l)}{\lambda} + \frac{\lambda df(x_0)(l)}{\lambda}\right] = \\
		df(x_0)(l) + \lim\limits_{\lambda\to 0}\frac{f(x_0+\lambda l) - f(x_0) - df(x_0)(\lambda l)}{\|\lambda l\|}.\|l\|.sgn(\lambda) = df(x_0)(l)
	\end{math}\\
	\halfbox{Следствие}\\
	\begin{math}
		\max{\frac{\partial f}{\partial l}(x_0) : \|l\| = 1} = \|grad f(x_0)\| \text{ и се достига за } l=\frac{grad f(x_0)}{\|grad f(x_0)\|} \\
		\left|\frac{\partial f}{\partial l}(x_0)\right| = \left|\langle gradf(x_0), l\rangle\right| \leq \|grad f(x_0)\|.\|l\|\\
		\min{\frac{\partial f}{\partial l}(x_0) : \|l\| = 1} = -\|grad f(x_0)\| \text{ и се достига за } l=-\frac{grad f(x_0)}{\|grad f(x_0)\|} \\
	\end{math}
	
	\begin{theorem}
		$U$ - отворено множество в $\mathbb{R}^n$ и $f:U\rightarrow \mathbb{R},\quad x_0\in U,\quad i\in\{1,...,n\}$,\\
		$\frac{\partial f}{\partial x_i}$ съществуват в $U$ и са непрекъснати в $x_0$. Тогава $f$ е диференцируема в $x_0$.
	\end{theorem}
	\begin{definition}
		Нека $U$ е отворено множество в $\mathbb{R}^n$, $f: U \rightarrow \mathbb{R}$. Тогава 
		$f$ се нарича \textbf{гладка} ($f\in C^1(U, \mathbb{R})$), ако $\frac{\partial f}{\partial x_i}$ съществуват и са непрекъснати в $U$ за всяко $i\in\{1,..,n\}$\\
	\end{definition}
	\halfbox{Следствие}\textbf{\textit{Гладките функции са диференцируеми}}\\
	\begin{math}
		\delta > 0, B_\delta(x_0) \in U, \quad h \in \mathbb{R}^n, \|h\| < \delta\\
		f(x_0+h)-f(x_0)=(f(x_0 + h_1e_1) - f(x_0)) + f(x_0 + h_1)
		\textasteriskcentered
	\end{math}
\end{document}
