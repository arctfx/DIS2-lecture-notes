\documentclass[12pt]{article}
\usepackage[a4paper, margin=2.5cm]{geometry}
\usepackage[T2A,T1]{fontenc}
\usepackage[utf8]{inputenc}
\usepackage[russian,english]{babel}

\usepackage{amsmath}
\usepackage{amsfonts}
\usepackage{graphicx}
\usepackage{epstopdf}
\usepackage{amssymb}
\usepackage{cancel}
\usepackage[utf8]{inputenc}
\DeclareUnicodeCharacter{25A9}{\dash}
\usepackage{color}
\definecolor{darkgray}{gray}{0.1}

\everymath{\displaystyle}

%Riemann integrals
\def\upint{\mathchoice%
	{\mkern13mu\overline{\vphantom{\intop}\mkern7mu}\mkern-20mu}%
	{\mkern7mu\overline{\vphantom{\intop}\mkern7mu}\mkern-14mu}%
	{\mkern7mu\overline{\vphantom{\intop}\mkern7mu}\mkern-14mu}%
	{\mkern7mu\overline{\vphantom{\intop}\mkern7mu}\mkern-14mu}%
	\int}
\def\lowint{
	\mkern3mu\underline{\vphantom{\intop}\mkern7mu}\mkern-10mu\int}
\newcommand{\RNum}[1]{\uppercase\expandafter{\romannumeral #1\relax}}
%Halfbox
\newcommand{\halfbox}[1]{\underline{\textbf{#1}:}\textbf{\large{| }}}
%Right cases bracket
\newenvironment{rcases}
{\left.\begin{aligned}}
	{\end{aligned}\right\rbrace}

%NEWTHEOREMS
\newtheorem{proposition}{Твърдение}
\newtheorem{definition}{Def.}
\newtheorem{lemma}{Lemma}
\newtheorem{theorem}{Th.}
\newcommand{\suma}[2]{\overset{#2}{\underset{#1}{\sum}}}
\newcommand{\spc}{\text{ }}

%BEGIN DOCUMENT
\begin{document}
	\selectlanguage{russian}
	\color{white}
	\pagecolor{darkgray}
	\title{Записки по ДИС2 - Лекция 7}
	\date{06.04.2023}
	\maketitle
	\begin{center}
		\Large
		\textbf{Безкрайни редове. Числови редове. Функционални редове.}
	\end{center}
	
	%###########################################################
	\section*{Редове с алтернативно сменящи се знаци}
	\begin{theorem}
		\textit{\textbf{Критерий на Лайбниц} за редове с алтернативно сменящи се знаци} \\
		$\sum_{n=1}^{\infty} (-1)^{n-1}a_n$, $\spc a_n \geq 0 \spc\forall n \in \mathbb{N}$ \\
		
		Ако $\{a_n\}_{n=1}^\infty$ е намаляваща (от някъде нататък) $a_n \xrightarrow[n \to \infty]{} 0$, то $\sum_{n=1}^{\infty}(-1)^{n-1} a_n$ е сходящ. \\
		$s_1 = a_1$ \\
		$s_2 = a_1 - a_2$ \\
		$s_3 = a_1 - a_2 + a_3$ \\
		$...........................$ \\
		
		$\{s_{2k-1}\}_{k=1}^{\infty}$ е намаляваща; $s_{2(k+1)-1} - s_{2k-1} = (-1)^{2k+\cancel{1}-\cancel{1}}$
		
	\end{theorem}
	
	\begin{definition}
		$\sum_{n=1}^{\infty}a_n$ се нарича \textbf{абсолютно сходящ}, ако $\sum_{n=1}^{\infty}|a_n|$ е сходящ.
	\end{definition}

	\begin{definition}
		$\sum_{n=1}^{\infty}a_n$ се нарича \textbf{условно сходящ}, ако е сходящ и $\sum_{n=1}^{\infty}|a_n|$ е разходящ.
	\end{definition}

	\begin{theorem}
		Необходимо и достатъчно условие на Коши за сходимост на числов ред.
	\end{theorem}

	\begin{proposition}
		Абсолютно сходящите редове са сходящи.
	\end{proposition}

	\halfbox{Комутативен закон}
	
	\begin{theorem}
		Ако $\sum_{n=1}^{\infty}a_n$ е абсолютно сходящ, то за него е в сила комутативния закон.
	\end{theorem}
	
	\begin{theorem}
		Теорема на Риман\\
		
	\end{theorem}
	
	\begin{theorem}
		content...
	\end{theorem}

	\begin{theorem}
		(Мертенс)\\
	\end{theorem}

	\section*{Редици и редове от функции.}
	\textit{To Be Continued In Lecture 8 ...}
\end{document}
