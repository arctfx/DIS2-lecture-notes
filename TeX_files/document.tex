\documentclass[12pt]{article}
\usepackage[a4paper, margin=2.5cm]{geometry}
\usepackage[T2A,T1]{fontenc}
\usepackage[utf8]{inputenc}
\usepackage[russian,english]{babel}

\usepackage{amsmath}
\usepackage{amsfonts}
\usepackage{graphicx}
\usepackage{epstopdf}
\usepackage{amssymb}
\usepackage[utf8]{inputenc}
\DeclareUnicodeCharacter{25A9}{\dash}
\usepackage{color}
\definecolor{darkgray}{gray}{0.1}

\newtheorem{proposition}{Твърдение}
\newtheorem{definition}{Def.}
\newtheorem{lemma}{Lemma}
\newtheorem{theorem}{Th.}
\newcommand{\suma}[2]{\overset{#2}{\underset{#1}{\sum}}}
\newcommand{\spc}{\text{ }}

%\newcommand{\lb}{\left(}
%\newcommand{\rb}{\right)}
%\newcommand{\lsb}{\left[}
%\newcommand{\rsb}{\right]}
%\newcommand{\lcb}{\left\{}
%\newcommand{\rcb}{\right\}}

\everymath{\displaystyle}

%Riemann integrals
\def\upint{\mathchoice%
	{\mkern13mu\overline{\vphantom{\intop}\mkern7mu}\mkern-20mu}%
	{\mkern7mu\overline{\vphantom{\intop}\mkern7mu}\mkern-14mu}%
	{\mkern7mu\overline{\vphantom{\intop}\mkern7mu}\mkern-14mu}%
	{\mkern7mu\overline{\vphantom{\intop}\mkern7mu}\mkern-14mu}%
	\int}
\def\lowint{
	\mkern3mu\underline{\vphantom{\intop}\mkern7mu}\mkern-10mu\int}

\begin{document}
	\selectlanguage{russian}
	\color{white}
	\pagecolor{darkgray}
	\title{Записки по ДИС2 - Лекция 1}
	\maketitle
	\begin{center}
		\Large
		\textbf{Малки и големи суми на Дарбу. Сума на Риман. Определен Интеграл на Риман. Критерий за интегруемост по Риман. Класове интегруеми функции.}
	\end{center}
	\textbf{Средно аритметично}
	$A=\frac{1}{n}\sum _{i=1}^{n}{a}_{i}$
	
	
	$A\left(f\right)=\frac{\underset{a}{\overset{b}{\int }}f\left(x\right)dx}{b-a} $ Средно аритметично на функция в интервал
	\\
	
	Нека функцията f е дефинирана в интервала $\left[a, b\right]$.
	
	$\tau : a=x_{0}<x_{1}<x_{2}<...<x_{n}=b : \left\{x_{i}\right\}$ - разбиване на $\left[a, b\right]$;
	
	$d\left(\tau\right)=\underset{1\leq i\leq n}{max}\left(x_{i} - x_{i-1}\right)$ - диаметър на разбиването;
	
	$\xi = \left\{\xi_{1}, \xi_{2}, \xi_{3}, ..., \xi _{n}\right\}$, $\xi_{i} \in \left[x_{i-1}, x_{i}\right],$ $i \in \left[1, n\right]$ - делящи/представителни/контролни точки.
	\\ 
	
	\textbf{Сума на Риман}
	
	$\sum _{i=1}^{n} f\left(\xi _{i}\right)\left(x_{i} - x_{i-1}\right)$
	$a_{x}$
	\\
	
	\textbf{Суми на Дарбу}
	
	$m_{i} := inf\left\{f\left(x\right): x\in \left[x_{i-1}, x_{i}\right]\right\}$
	
	$M_{i} := sup\left\{f\left(x\right): x\in \left[x_{i-1}, x_{i}\right]\right\}$
	
	
	Малка сума на Дарбу 
	$s_{f}\left(\tau\right) := \sum _{i=1}^{n} m_{i}\left(x_{i}-x_{i-1}\right)$
	
	Голяма сума на Дарбу
	$S_{f}\left(\tau\right) := \sum _{i=1}^{n} M_{i}\left(x_{i}-x_{i-1}\right)$
	\\
	
	$f: \mathbb{R}^{n} \mapsto \mathbb{R}$
	% //////////////////////////////////////////////////////////
	% //////////////////////////////////////////////////////////
	% //////////////////////////////////////////////////////////
	%LEMMA1
	\begin{lemma}
		Нека $\tau^{*}$, $\tau$ са подразбивания на интервала $\left[a, b\right]$, такива че  $\tau^{*} \geq \tau$. 
		$\left(\text{тогава ще казваме, че }  \tau^{*} \textbf{ е по-фино от } \tau\right)$. 
		Ако $\tau^{*} \geq \tau$, то $S_{f}\left(\tau^{*}\right) \geq S_{f}\left(\tau\right)$ и $s_{f}\left(\tau^{*}\right) \geq S_{f}\left(\tau\right)$.
		\\
	\end{lemma}
	
	% BEGIN PROOF
	\textbf{Доказателство}: (Б.О.О.) $\tau^{*}$ се получава от $\tau$ с прибавянето на една точка.
	Нека $\tau : a=x_{0}<x_{1}<...<x_{n}=b$ и
	$\tau^{*} : a=x_{0}<x_{1}<...<x_{i-1}<x^{*}<x_{i}<...<x_{n}=b$ 
	\newline
	
	\begin{equation*}
		\begin{aligned}
			&\spc S_{f}\left(\tau\right) - S_{f}\left(\tau^{*}\right) =
			\\
			&\spc\suma{i=1}{n}\spc\underset{\left[x_{j-1}, x_{j}\right]}{sup f} \left(x_{j} - x_{j-1}\right)\spc-
			\\
			&\left(
			\suma{j=1}{i-1}\underset{\left[x_{j-1}, x_{j}\right]}{sup f} \left(x_{j} - x_{j-1}\right) + 
			\underset{\left[x_{i-1},\spc x^{*}\right]}{sup f} \left(x^{*} - x_{i-1}\right) +
			\right. \\ &\left.
			\underset{\left[x^{*},\spc x_{i}\right]}{sup f} \left(x_{i} - x^{*}\right) +
			\suma{j=i+1}{n}\underset{\left[x_{j-1}, x_{j}\right]}{sup f} \left(x_{j} - x_{j-1}\right)
			\right) =
			\\
			&\underset{\left[x_{i-1},\spc x_{i}\right]}{sup f} \left(x_{i} - x_{i-1}\right) - 
			\underset{\left[x_{i-1},\spc x^{*}\right]}{sup f} \left(x_{*} - x_{i-1}\right) -
			\underset{\left[x^{*},\spc x_{i-1}\right]}{sup f} \left(x_{i} - x_{*}\right) \geq
			\\
			&\underset{\left[x_{i-1},\spc x_{i}\right]}{sup f} \left(x_{i} - x_{i-1}\right) - 
			\underset{\left[x_{i-1},\spc x_{i}\right]}{sup f} \left(x^{*} - x_{i-1}\right) -
			\underset{\left[x_{i-1},\spc x_{i}\right]}{sup f} \left(x_{i} - x^{*}\right) = 0 
		\end{aligned}
	\end{equation*}
	\begin{flushright}
		$\blacksquare$
	\end{flushright}
	Аналогично се пресмята за малките суми на Дарбу ......
	% END PROOF
	
	%LEMMA2
	\begin{lemma}
		Некa $\tau_{1} и \tau_{2}$ са произволни подразбивания на $\left[a, b\right]$.\newline
		Тогава $s_{f}\left(\tau_{1}\right) \leq S_{f}\left(\tau_{2}\right)$.
	\end{lemma}
	% BEGIN PROOF
	\textbf{Доказателство}: Нека $\tau^{*}$ е по-фино от $\tau_{1}$ и от $\tau_{2}$. Очевидно $\tau^{*}$ съществува и може да се получи като обединение на точките от $\tau_{1}$ и $\tau_{2}$. От \underline{Lemma 1} 
	имаме:
	\begin{center}
		$s_{f}\left(\tau_{1}\right) \leq s_{f}\left(\tau^{*}\right)$; \\
		$S_{f}\left(\tau^{*}\right) \leq S_{f}\left(\tau_{2}\right)$.
	\end{center}
	Нека $f: \left[a, b\right] \mapsto \mathbb{R}$ е ограничена.
	......
	% END PROOF
	
	\begin{definition}
		Нека $f: \left[a, b\right] \mapsto \mathbb{R}$ е ограничена. Казваме, че f е интегруема по Риман, ако
		$\upint_a^b f(x)\,\mathrm{d}x = \lowint_a^b f(x)\,\mathrm{d}x$, 
		което се нарича \textbf{риманов интеграл} на $f$ в $\left[a,\spc b\right]$ и се означава с $\int_{a}^{b} f$ или $\int_{a}^{b} f(x)\,\mathrm{d}x$, $\int_{a}^{b} f(w)\,\mathrm{d}w$ etc.
	\end{definition}

	% TO-DO Add example
	
	
	\begin{theorem}
		\textbf{(Критерий за интегруемост по Риман)} 	Нека $f:\left[a, b\right] \rightarrow \mathbb{R}$ е ограничена.
		Твърдим, че f е интегруема по Риман тогава и само тогава, когато за всяко епсилон по-голямо от нула съществуват подразбивания $\tau_{1} и \tau_{2}$ на $\left[a, b\right]$, за които $S_{f}\left(\tau_{1}\right) - s_{f}\left(\tau_{2}\right) < \varepsilon$. Еквивалентно, за всяко епсилон по-голямо от нула съществува подразбиване на $\left[a, b\right]$, за което $S_{f}\left(\tau\right) - s_{f}\left(\tau\right) < \varepsilon$.
	\end{theorem}
	% BEGIN PROOF
	\textbf{Доказателство}: ($\Rightarrow$) Нека $\varepsilon>0$ е произволно. Имаме:
	\[
	\int_{a}^{b} f + \frac{\varepsilon}{2}\spc = \upint_a^b f + \frac{\varepsilon}{2}\spc
	> \upint_a^b f \Rightarrow 
	\exists\tau_{1} \text{ - подразбиване на }[a,b],\spc
	S_{f}(\tau_{1})<\int_{a}^{b} f + \frac{\varepsilon}{2}
	\]
	\[
	\int_{a}^{b} f - \frac{\varepsilon}{2}\spc = \lowint_a^b f - \frac{\varepsilon}{2}\spc
	< \lowint_a^b f \Rightarrow 
	\exists\tau_{2} \text{ - подразбиване на }[a,b],\spc
	s_{f}(\tau_{2})>\int_{a}^{b} f - \frac{\varepsilon}{2}
	\]
	\[
	S_f(\tau_{1})-s_f(\tau_{2}) < \left(\int_{a}^{b}f + \frac{\varepsilon}{2}\right) - \left(\int_{a}^{b}f - \frac{\varepsilon}{2}\right) = \frac{\varepsilon}{2} + \frac{\varepsilon}{2} = \varepsilon
	\]
	($\Leftarrow$) Допускаме противното:\\
	$\Rightarrow \forall\tau_{1},\tau_{2} : S_f(\tau_{1})-s_f(\tau_{2}) \geq \upint_{a}^{b}f - \lowint_{a}^{b}f > 0 $ - противоречие.\\
	($\Uparrow$) Достатъчно е да положим $\tau_{1} := \tau$ и $\tau_{1} := \tau$\\
	($\Downarrow$) Нека $\varepsilon>0$. Имаме разбиванията $\tau_{1}, \tau_{2}$, за които $S_f(\tau_{1}) - s_f(\tau_{2}) < \varepsilon$. Нека $\tau$ е разбиване по-фино от $\tau_{1}$ и $\tau_{2}$, т.е. $\tau\geq\tau_1, \tau\geq\tau_2$\\
	$\Rightarrow S_f(\tau) - s_f(\tau) \leq S_f(\tau_1) - s_f(\tau_2) < \varepsilon$\\
	$\blacksquare$
	% END PROOF
	
	\begin{definition}
		\textbf{Осцилация на функция:} Нека $f:[a,b]\rightarrow\mathbb{R}$ е ограничена. Осцилация на $f$ в $[a,b]$ дефинираме като:\\
		\[\omega(f;\spc[a,b]) := sup\{|f(x) - f(y)| : x,y\in [a,b]\}.\]
	\end{definition}
	
	\begin{lemma}
		Осцилацията на функцията f в даден интервал е равна на разликата между супремума и инфимума на функцията в дадения интервал.
		\[\omega(f;\spc[a,b]) = \underset{\left[a, b\right]}{sup} f - \underset{\left[a, b\right]}{inf} f\]
	\end{lemma}
	%PROOF BEGIN
	%PROOF END
%
%+
%

	\begin{proposition}
		Непрекъснатите функции са интегруеми.
	\end{proposition}
	% BEGIN PROOF
	\textbf{Доказателство}: Нека $f:[a,b] \rightarrow \mathbb{R}$ - непрекъсната. Oт Теорема на Вайерщрас следва, че $f$ е ограничена.\\
	Tеорема на Кантор $ \Rightarrow \exists \delta>0\spc\forall {x}^{\prime}, {x}^{\prime\prime} \in [a,b],\spc |{x}^{\prime} - {x}^{\prime\prime}|<\delta : |f({x}^{\prime}) - f({x}^{\prime\prime})| < \varepsilon $ 
	Взимаме $\tau : a=x_{0}<x_{1}<x_{2}<...<x_{n}=b$, за което\\ $d(\tau):=\max{\{x_i - x_{i-1} : i\in\{1,...,n\}\}} < \delta$,\\
	Тогава имаме:\\
	$0 \leq S_f(\tau) - s_f(\tau) = \suma{i=1}{n}\omega(f; [x_{i-1}, x_i])(x_i - x_{i-1}) \leq \varepsilon \suma{i=1}{n}(x_i - x_{i-1}) = \varepsilon(b-a)$\\
	След граничен преход получаваме:\\
	$\lim_{d(\tau) \to 0}\suma{i=1}{n}w_i(f)\Delta x_i = 0$ и следователно $f$ е интегруема в $[a,b]$. $\blacksquare$
	%END PROOF

%
%
%

	\begin{proposition}
		Нека f е ограничена и има краен брой точки на прекъсване. Тогава f е интегруема.
	\end{proposition}

	\begin{proposition}
		Монотонните функции са интегруеми.
	\end{proposition}

	\begin{proposition}
		Нека $f:\left[a, b\right] \rightarrow \mathbb{R}$ е ограничена и имаме произволно разбиване на интервала, в който е дефинирана.
		%\textbackslash{}left[a,\textbackslash{}spc b\textbackslash{}right]
		$\tau : a=x_{0}<x_{1}<x_{2}<...<x_{n}=b$.
		Тогава $s_{f}\left(\tau\right)\spc=\spc inf\{\sigma_{f}\left(\tau,\spc\xi\right) : \xi\spc\text{- представителна точка}\}$;
		$S_{f}\left(\tau\right)\spc=\spc sup\{\sigma_{f}\left(\tau,\spc\xi\right) : \xi\spc\text{- представителна точка}\}$.
	\end{proposition}
	\textbf{Доказателство:} ......
	
	\begin{definition}
		Нека $d\left(\tau\right) := max\{x_{i}-x_{i-1} : i\in\{1,...,n\}\}$ и $\tau : a=x_{0}<...<x_{n}=b$. Казваме, че сумите на Риман за $f: \left[a, b\right] \rightarrow \mathbb{R}$ имат граница $I\in\mathbf{R}$, когато $d\left(\tau\right)$ клони към 0, ако за всяко $\varepsilon\geq 0$ съществува $\delta\geq 0$, такова че за всеки избор на подразбиване $\tau$ на $\left[a, b\right]$ с $d\left(\tau\right)\leq\delta$ и за всеки избор на предствавителна точка е в сила $|\sigma_{f}\left(\tau;\spc\xi\right)-I|\leq\varepsilon$.
	\end{definition}
	
	\begin{lemma}
		Нека $\tau : a=x_{0}<x_{1}<x_{2}<...<x_{n}=b$ е подразбиване на $\left[a, b\right]$. Нека $\tau^{*}\geq\tau$, $\tau^{*}$ се получава от $\tau$ чрез прибавянето на k точки. Тогава $0\leq S_{f}\left(f\right)-S_{f}\left(\tau^{*}\right)\leq...$.
		Нека $m := \underset{\left[a, b\right]}{inf} f$ и $М := \underset{\left[a, b\right]}{sup} f$
	\end{lemma}
	
	
	\begin{center}
		\textbf{Основни свойства на интеграла}
	\end{center}
	(I) Линейност
	
	%▩
\end{document}
