\documentclass[12pt]{article}
\usepackage[a4paper, margin=3cm]{geometry}
\usepackage[T2A,T1]{fontenc}
\usepackage[utf8]{inputenc}
\usepackage[russian,english]{babel}

\usepackage{amsmath}
\usepackage{amsfonts}
\usepackage{graphicx}
\usepackage{epstopdf}
\usepackage{amssymb}
\usepackage[utf8]{inputenc}
\DeclareUnicodeCharacter{25A9}{\dash}
\usepackage{color}
\definecolor{darkgray}{gray}{0.1}

\newtheorem{proposition}{Твърдение}
\newtheorem{definition}{Def.}
\newtheorem{lemma}{Lemma}
\newtheorem{theorem}{Th.}
\newcommand{\suma}[2]{\overset{#2}{\underset{#1}{\sum}}}
\newcommand{\spc}{\text{ }}

\everymath{\displaystyle}

%Riemann integrals
\def\upint{\mathchoice%
	{\mkern13mu\overline{\vphantom{\intop}\mkern7mu}\mkern-20mu}%
	{\mkern7mu\overline{\vphantom{\intop}\mkern7mu}\mkern-14mu}%
	{\mkern7mu\overline{\vphantom{\intop}\mkern7mu}\mkern-14mu}%
	{\mkern7mu\overline{\vphantom{\intop}\mkern7mu}\mkern-14mu}%
	\int}
\def\lowint{
	\mkern3mu\underline{\vphantom{\intop}\mkern7mu}\mkern-10mu\int}

\begin{document}
	\selectlanguage{russian}
	\color{white}
	\pagecolor{darkgray}
	\title{Записки по ДИС2 - Лекция 2}
	\maketitle
	\begin{center}
		\Large
		\textbf{Класове интегруеми функции. }
	\end{center}

	\begin{theorem}
		Нека $f: \left[a, b\right] \rightarrow \mathbb{R}$ и нека съществува  $\lim_{x\to\infty} \sigma_{f}\left(\tau;\spc\xi\right) = I$. Тогава f е ограничена, f е интегруема по Риман и $\int_{a}^{b} f = I$.
		
	\end{theorem}
	\textbf{Доказателство:} Нека $f: \left[a, b\right] \rightarrow \mathbb{R}$ и $\epsilon > 0$ e достатъчно голямо. Нека $\epsilon = 3 > 0$ е фиксирано, oткъдето следва: \\
	$\exists\delta>0\forall\tau, d(\tau)<\delta\forall\xi - представителни точки за \tau : |\sigma_{f}\left(\tau,\spc\xi\right)-I|<3$.\\
	Нека $\tau : a=x_{0}<x_{1}<x_{2}<...<x_{n}=b$ е такова, че $d(\tau)<\delta$.
	Тогава за всеки избор на $\xi_{i}\in\left[x_{i-1},\spc x_{i}\right],\spc i\in\{1,...,n\}$ е в сила\\
	\[I-3 < \sum _{i=1}^{n}f(\xi_{i})(x_{i}-x_{i-1})<I+3\]
	Достатъчно е да докажем, че $f$ е ограничена в $\left[x_{i-1},\spc x_{i}\right]$.\\
	Фиксираме $\xi_{j}\in\left[x_{j-1},\spc x_{j}\right],\spc j\neq i$.\\
	........
	
	\begin{theorem}
		Ако $f: \left[a, b\right] \rightarrow \mathbb{R}$ е интегруема по Риман (следователно и $f$ е ограничена), то съществува $\lim_{x\to\infty} \sigma_{f}\left(\tau;\spc\xi\right) = \int_{a}^{b} f$.
		
	\end{theorem}

\end{document}