\documentclass[12pt]{article}
\usepackage[a4paper, margin=2.5cm]{geometry}
\usepackage[T2A,T1]{fontenc}
\usepackage[utf8]{inputenc}
\usepackage[russian,english]{babel}

\usepackage{tikz}
\usepackage{tcolorbox}
\usepackage{amsfonts}
\usepackage{mathtools}
\usepackage{graphicx}
\usepackage{epstopdf}
\usepackage{amssymb}
\usepackage{cancel}
\usepackage{hf-tikz}
\usepackage{pgfplots}
\usepackage[utf8]{inputenc}
\DeclareUnicodeCharacter{25A9}{\dash}
\usepackage{color}
\definecolor{darkgray}{gray}{0.1}

\everymath{\displaystyle}

%Riemann integrals
\def\upint{\mathchoice%
	{\mkern13mu\overline{\vphantom{\intop}\mkern7mu}\mkern-20mu}%
	{\mkern7mu\overline{\vphantom{\intop}\mkern7mu}\mkern-14mu}%
	{\mkern7mu\overline{\vphantom{\intop}\mkern7mu}\mkern-14mu}%
	{\mkern7mu\overline{\vphantom{\intop}\mkern7mu}\mkern-14mu}%
	\int}
\def\lowint{
	\mkern3mu\underline{\vphantom{\intop}\mkern7mu}\mkern-10mu\int}
\newcommand{\RNum}[1]{\uppercase\expandafter{\romannumeral #1\relax}}
%Halfbox
\newcommand{\halfbox}[1]{\underline{\textbf{#1}:}\textbf{\large{| }}}

%NEWTHEOREMS
\newtheorem{proposition}{Твърдение}
\newtheorem{definition}{Def.}
\newtheorem{lemma}{Lemma}
\newtheorem{theorem}{Th.}
\newcommand{\suma}[2]{\overset{#2}{\underset{#1}{\sum}}}
\newcommand{\spc}{\text{ }}

%BEGIN DOCUMENT
\begin{document}
	\selectlanguage{russian}
	\color{white}
	\pagecolor{darkgray}
	\title{Записки по ДИС2 - Лекция 5}
	\date{23.03.2023}
	\maketitle
	\begin{center}
		\Large
		\textbf{Несобствени интеграли. Елементарни свойства. Несобствени интеграли с неотрицателни подинтегрални функции.}
	\end{center}
	
	%###########################################################
	Нека разгледаме определения интеграл $а \int_{0}^{1}\frac{1}{\sqrt{1-x^2}}\,\mathrm{d}x$. Подинтегралната функция не е дефинирана в ограниченият интервал $[a,b]$ и затова не можем да определим стойността на интеграла с досегашните ни знания. Но имаме добре дефиниран интеграла $\int_{0}^{1-\varepsilon}\frac{1}{\sqrt{1-x^2}}\,\mathrm{d}x = arcsin{\big|}_{0}^{1-\varepsilon} = arcsin(1-\varepsilon)$. След граничен преход получаваме:\\
	\[\lim_{\substack{\varepsilon\to 0 \\ \varepsilon>0}}\int_{0}^{1-\varepsilon}\frac{1}{\sqrt{1-x^2}}\,\mathrm{d}x = \lim_{\substack{\varepsilon\to 0 \\ \varepsilon>0}}(arcsin(1-\varepsilon)) = \frac{\pi}{2}\]
	
	\begin{definition} \textbf{\textit{Несобствен интеграл от \RNum{1}-ви род}}:
		Нека $f : [a,+\infty) \rightarrow \mathbb{R}$ е интегруема (аналогично за $f : (-\infty, a] \rightarrow \mathbb{R}$) във всеки интервал от вида $[a,p]$,където $p\geq a$. Ако съществува границата
		$\lim_{p\to\infty}\int_{a}^{p}f(t)\,\mathrm{d}t$ казваме, че $\int_{a}^{+\infty}f(t)\,\mathrm{d}t$ е сходящ и пишем $\int_{a}^{+\infty}f(t)\,\mathrm{d}t = \lim_{p\to\infty}\int_{a}^{p}f(t)\,\mathrm{d}t$. Ако границата не съществува, интегралът се нарича разсходящ.
	\end{definition}

	\begin{definition} \textbf{\textit{Несобствен интеграл от \RNum{2}-ри род}}:
		Нека $f : [a,b) \rightarrow \mathbb{R}$ е интегруема във всеки интеграл от $[a,p],\spc p\in[a,b)$. Ако съществува границата $\lim_{\substack{p\to b \\ p<b}}\int_{a}^{p}f(t)\,\mathrm{d}t$, интегралът $\int_{a}^{b}f(t)\,\mathrm{d}t$ се нарича сходящ и $\int_{a}^{b}f(t)\,\mathrm{d}t := \lim_{\substack{p\to b \\ p<b}}\int_{a}^{p}f(t)\,\mathrm{d}t$, наричаме го интеграл от \RNum{2}-ри с особеност в $b$.
	\end{definition}
	
	Аналогично на \textbf{Def. 1} можем да дефинираме несобствен интеграл от \RNum{1}-ви род с особеност в $-\infty$:
	\[\int_{-\infty}^{a}f(t)\,\mathrm{d}t := \lim_{p\to -\infty}\int_{p}^{a}f(t)\,\mathrm{d}t\] 
	С особеност в долния край на интервала:
	\[\int_{a}^{b}f(t)\,\mathrm{d}t := \lim_{\substack{p\to a \\ p>a}}\int_{p}^{b}f(t)\,\mathrm{d}t\]\\
	
	\begin{center}
		\textbf{\large\underline{Елементарни свойства на несобствените интеграли}}
	\end{center}
	
	\textbf{(\RNum{1}) Линейност}\\
	\[
	\begin{aligned}
		\text{Нека}\spc &\int_{a}^{+\infty}f(t)\,\mathrm{d}t, \int_{a}^{+\infty}g(t)\,\mathrm{d}t \spc\text{ са сходящи} \Rightarrow \\
		&\int_{a}^{+\infty}f(t) + g(t)\,\mathrm{d}t = \int_{a}^{+\infty}f(t)\,\mathrm{d}t + \int_{a}^{+\infty}g(t)\,\mathrm{d}t\text{;}\\
		&\lambda\in\mathbb{R}:
		\int_{a}^{+\infty}\lambda f(t)\,\mathrm{d}t = \lambda\int_{a}^{+\infty}f(t)\,\mathrm{d}t\spc\text{ са също сходящи.}
	\end{aligned}
	\]
	
	\textbf{(\RNum{2}) Смяна на променливите}\\
	При смяна на променливите можем да минем от собствен в несобствен интеграл и обратно. Тогава ако собственият е сходящ, то несобствения ще е сходящ. Теоремата за смяна на променливите си остава.\\
	
	\textbf{(\RNum3)}\\
	\[
	\begin{aligned}
		c>a: \int_{a}^{+\infty}f(t)\,\mathrm{d}t \spc\text{- сходящ }\Rightarrow\int_{c}^{+\infty}f(t)\,\mathrm{d}t \spc\text{- сходящ}
	\end{aligned}
	\]
	
	\textbf{(\RNum4) Повече от една особености}\\
	Нека $f$ е непрекъсната, тогава интегралът отляво е сходящ, ако и двата отдясно са сходящи:
	\[
	\begin{aligned}
		\int_{-\infty}^{+\infty}f(t)\,\mathrm{d}t = \int_{-\infty}^{a}f(t)\,\mathrm{d}t + \int_{a}^{+\infty}f(t)\,\mathrm{d}t
	\end{aligned}
	\]
	\\
	
	\halfbox{Пример} $\int_{0}^{+\infty}\frac{\mathrm{d}t}{\mathrm{ln}t}\spc = \spc\int_{0}^{\frac{1}{2}}\frac{\mathrm{d}t}{\mathrm{ln}t} + \int_{\frac{1}{2}}^{1}\frac{\mathrm{d}t}{\mathrm{ln}t} + \int_{1}^{2}\frac{\mathrm{d}t}{\mathrm{ln}t} + \int_{2}^{+\infty}\frac{\mathrm{d}t}{\mathrm{ln}t}$\\
	
	\begin{center}
		\textbf{\large\underline{Несобствени интеграли с неотрицателни подинтегрални функции}}
	\end{center}
	$\int_{a}^{+\infty}f(x)\,\mathrm{d}x, \spc \int_{a}^{b}f(x)\,\mathrm{d}x, \spc f \geq 0$
	
	\begin{equation*}
		\begin{rcases}
			F(p) &= \int_{a}^{p}f(x)\,\mathrm{d} \\
			f \geq 0 &\text{ в }[a,+\infty)  \\
		\end{rcases}
		\Rightarrow F\text{ - растяща в }[a,+\infty)
	\end{equation*}

	\[
	F(p_{2}) - F(p_{1}) = \int_{a}^{p_2}f(x)\,\mathrm{d}x - \int_{a}^{p_1}f(x)\,\mathrm{d}x
	\]
	
	\begin{lemma}
		Нека $F:[a,+\infty) \rightarrow \mathbb{R}$ - растяща. Тогава:\\
		\[
		\lim\limits_{p\rightarrow +\infty}F(p) \;\text{съществува} \Leftrightarrow F \text{ e ограничена отгоре в } [a,+\infty)
		\]
	\end{lemma}
	%BEGIN PROOF
	\textbf{\underline{Доказателство:}}\\
	$\begin{aligned}
		(\Rightarrow)\ &l = \lim\limits_{p\rightarrow +\infty}F(p) \ \Rightarrow \ \exists p_{0} \geq a \spc \forall p \geq p_{0}: F(p)\in(l-1,l+1)
		\Rightarrow \ F(p) < l+1 \ \forall p \in [a, +\infty)\\
		&a\leq p \leq p_{0} \rightarrow F(p) \leq F(p_{0}) < l+1\\
		&p\geq p_{0} \rightarrow F(p) < l+1
	\end{aligned}$
	\\$\spc$
	\\
	$\begin{aligned}
		(\Leftarrow)\ &\mathbb{R}\ni l = \sup\{F(p):p\geq a\}\\
		&\boxed{\varepsilon>0} \quad l+\varepsilon > \underbracket[0.10ex]{ l > l - \varepsilon}{} \Rightarrow \exists p_0 \geq a, \spc F(p_0) > l-\varepsilon \ 
		quad (\text{от дефиниция на супремум})
	\end{aligned}$
	%END PROOF
	
\end{document}
