\documentclass[12pt]{article}
\usepackage[a4paper, margin=2.5cm]{geometry}
\usepackage[T2A,T1]{fontenc}
\usepackage[utf8]{inputenc}
\usepackage[russian,english]{babel}

\usepackage{amsmath}
\usepackage{amsfonts}
\usepackage{graphicx}
\usepackage{epstopdf}
\usepackage{amssymb}
\usepackage{cancel}
\usepackage[utf8]{inputenc}
\DeclareUnicodeCharacter{25A9}{\dash}
\usepackage{color}
\definecolor{darkgray}{gray}{0.1}

\everymath{\displaystyle}

%Riemann integrals
\def\upint{\mathchoice%
	{\mkern13mu\overline{\vphantom{\intop}\mkern7mu}\mkern-20mu}%
	{\mkern7mu\overline{\vphantom{\intop}\mkern7mu}\mkern-14mu}%
	{\mkern7mu\overline{\vphantom{\intop}\mkern7mu}\mkern-14mu}%
	{\mkern7mu\overline{\vphantom{\intop}\mkern7mu}\mkern-14mu}%
	\int}
\def\lowint{
	\mkern3mu\underline{\vphantom{\intop}\mkern7mu}\mkern-10mu\int}
\newcommand{\RNum}[1]{\uppercase\expandafter{\romannumeral #1\relax}}
%Halfbox
\newcommand{\halfbox}[1]{\underline{\textbf{#1}:}\textbf{\large{| }}}
%Right cases bracket
\newenvironment{rcases}
{\left.\begin{aligned}}
	{\end{aligned}\right\rbrace}

%NEWTHEOREMS
\newtheorem{proposition}{Твърдение}
\newtheorem{definition}{Def.}
\newtheorem{lemma}{Lemma}
\newtheorem{theorem}{Th.}
\newcommand{\suma}[2]{\overset{#2}{\underset{#1}{\sum}}}
\newcommand{\spc}{\text{ }}

%BEGIN DOCUMENT
\begin{document}
	\selectlanguage{russian}
	\color{white}
	\pagecolor{darkgray}
	\title{Записки по ДИС2 - Лекция 6}
	\date{30.03.2023}
	\maketitle
	\begin{center}
		\Large
		\textbf{Числови редове. Функционални редове.}
	\end{center}
	
	%###########################################################
	\begin{definition}
		\textbf{Числов ред}
	\end{definition}

	\begin{definition}
		\textbf{Сходящ ред}
	\end{definition}
	
	\halfbox{Пример} \textbf{\underline{Геометрична прогресия}}
	
	\halfbox{Пример}
	
	\halfbox{Пример} \textbf{\underline{Хармоничен ред}}
	
	
	\section*{Елементарни свойства}
	I) \halfbox{НУ сходимост}
	\\
	II) \halfbox{Линейност}
	\\
	III) При прибавяне на краен брой членове, сходимостта не се променя, но сумата се.\\
	\halfbox{Свойство} Сходимостта (или разходимостта) се запазва при прибавянето или задраскването на краен брой членове.
	\\
	IV) \halfbox{Твърдение} 
	
	\section*{Редове с неотрицателен общ член}
	$\sum_{n=1}^{\infty}a_n,\quad a_n\geq 0$
	
	\halfbox{Принцип за сравнение}
	
	\halfbox{Интегрален критерий} (\textit{Коши-Маклорен})
	\underline{Доказателство}:
	
	\halfbox{Критерий на Даламбер} (ДУ сходимост + ДУ разходимост)
	
	\halfbox{Следствие}
	
	\halfbox{Пример} Експонентата надделява над полинома
	
	\halfbox{Пример}
	
	\halfbox{Критерий на Коши}
	
	\halfbox{Следствие}
	
	\halfbox{Пример}
	
	\halfbox{Задача} ДСД сходимост по Черазо (домашно за вкъщи)
	\boxed{a_n>0} Ако $\exists\lim\limits_{n\to\infty}\frac{a_n+1}{a_n}=l$, то $\lim\limits_{n\to\infty}\sqrt[n]{a_n}=l$
	
	Критерият на Даламбер е удобен при двойни факториели. Критерият на Коши прилагаме при n-ти корени.
	
	
\end{document}
