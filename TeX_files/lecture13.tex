\documentclass[12pt]{article}
\usepackage[a4paper, margin=2.5cm]{geometry}
\usepackage[T2A,T1]{fontenc}
\usepackage[utf8]{inputenc}
\usepackage[russian,english]{babel}

\usepackage{wrapfig}
\usepackage{amsmath}
\usepackage[customcolors]{hf-tikz}
\usepackage{tikz}
\usetikzlibrary{patterns}
\usetikzlibrary {patterns.meta}
\usepackage{tcolorbox}
\usepackage{amsfonts}
\usepackage{mathtools}
\usepackage{graphicx}
\usepackage{epstopdf}
\usepackage{amssymb}
\usepackage{cancel}
\usepackage{hf-tikz}
\usepackage{pgfplots}
\usepackage[utf8]{inputenc}\usepackage{wrapfig}
\usepackage{amsmath}
\usepackage[customcolors]{hf-tikz}
\usepackage{tikz}
\usetikzlibrary{patterns}
\usetikzlibrary {patterns.meta}
\usepackage{tcolorbox}
\usepackage{amsfonts}
\usepackage{mathtools}
\usepackage{graphicx}
\usepackage{epstopdf}
\usepackage{amssymb}
\usepackage{cancel}
\usepackage{hf-tikz}
\usepackage{pgfplots}
\usetikzlibrary{shadings}
\usepackage[utf8]{inputenc}
\DeclareUnicodeCharacter{25A9}{\dash}
\usepackage{color}
\definecolor{darkgray}{gray}{0.1}

\everymath{\displaystyle}

%Riemann integrals
\def\upint{\mathchoice%
	{\mkern13mu\overline{\vphantom{\intop}\mkern7mu}\mkern-20mu}%
	{\mkern7mu\overline{\vphantom{\intop}\mkern7mu}\mkern-14mu}%
	{\mkern7mu\overline{\vphantom{\intop}\mkern7mu}\mkern-14mu}%
	{\mkern7mu\overline{\vphantom{\intop}\mkern7mu}\mkern-14mu}%
	\int}
\def\lowint{
	\mkern3mu\underline{\vphantom{\intop}\mkern7mu}\mkern-10mu\int}
\newcommand{\RNum}[1]{\uppercase\expandafter{\romannumeral #1\relax}}
%Halfbox
\newcommand{\halfbox}[1]{\underline{\textbf{#1}:}\textbf{\large{| }}}


%NEWTHEOREMS
\newtheorem{proposition}{Твърдение}
\newtheorem{definition}{Def.}
\newtheorem{lemma}{Lemma}
\newtheorem{theorem}{Th.}
\newcommand{\suma}[2]{\overset{#2}{\underset{#1}{\sum}}}
\newcommand{\spc}{\text{ }}

%BEGIN DOCUMENT
\begin{document}
	\selectlanguage{russian}
	\color{white}
	\pagecolor{darkgray}
	\title{Записки по ДИС2 - Лекция 13}
	\date{25.05.2023}
	\maketitle
	\begin{center}
		\Large
		\textbf{Диференциране на композиция.}
	\end{center}
	
	%###########################################################
	
	\section*{Преговор}
	\subsection*{Функции от вида $\mathbb{R}^n \rightarrow \mathbb{R}$}
	\begin{math}
		f: U \rightarrow \mathbb{R}, \spc U \subset \mathbb{R}^n - \text{ отворено}, \spc x_0\in U, \spc
		df(x_0): \mathbb{R}^n \rightarrow \mathbb{R} - \text{ линеен оператор такъв, че: }\\
		f(x) = f(x_0) + df(x_0)(x-x_0) + \varphi(x, x_0) \text{ и } \frac{\varphi(x, x_0)}{\|x-x_0\|}\underset{x\to x_0}{\longrightarrow}0.
	\end{math}
	От предната лекция имаме следните твърдения:\\
	\begin{proposition}
		Ако $f$ е диференцируема в $x_0$, то частните производни $\frac{\partial f}{\partial x_i}(x_0) = \lim\limits_{\lambda \to 0}\frac{f(x_0 + \lambda e_i) - f(x_0)}{\lambda}$, $i\in\{1,..,n\}$ съществуват и $\frac{\partial f}{\partial x_i}(x_0) = df(x_0)(e_i)$.
	\end{proposition}
	$\nabla f(x_0) = \left(\frac{\partial f}{\partial x_1}(x_0), ..., \frac{\partial f}{\partial x_n}(x_0)\right)$
	$\quad df(x_0)(h) = \langle \nabla f(x_0), h\rangle$ 
	\begin{proposition}
		Ако частните производни $\frac{\partial f}{\partial x_i}(x_0)$ съществуват в $U$ и са непрекъснати в $x_0$, то $f$ е диференцируема в $x_0$.
	\end{proposition}
	
	\subsection*{Функции от вида $\mathbb{R}^m \rightarrow \mathbb{R}^n$}
	\begin{math}
		f: U \rightarrow \mathbb{R}^m, \spc U\subset \mathbb{R} - \text{ отворено}, \spc x_0\in U,
		\spc f(x) = \begin{pmatrix}
						f_1(x) \\
						\vdots \\
						f_m(x)
					\end{pmatrix}
	\end{math}
	\begin{proposition}
		$f$ е диференцируема в $x_0$, ако $f(x) = f(x_0) + df(x_0)(x-x_0) + o(\|x-x_0\|)$, където $ df(x_0):\mathbb{R}^m \rightarrow \mathbb{R}^n$ - линеен оператор $\Leftrightarrow f_i \text{ е диференцируема в } x_0 \text{ за всяко } i = \overline{1, m}$. Тоест:\\
		\begin{math}
		df(x_0)(h) =
			\begin{pmatrix}
				\nabla f_1(x_0) \\
				\vdots \\
				\nabla f_m(x_0)
			\end{pmatrix} h = f'(x_0)h, \text{ т.е. } \quad
		f'(x_0) = 
			\begin{pmatrix}
				\frac{\partial f_1}{\partial x_1}(x_0)
				\spc ..... \spc
				\frac{\partial f_1}{\partial x_n}(x_0)
				\\
				..... \quad ..... \quad .....
				\\
				\frac{\partial f_m}{\partial x_1}(x_0)
				\spc ..... \spc
				\frac{\partial f_m}{\partial x_n}(x_0)
			\end{pmatrix}_{(n \times m)}
		\end{math}
	\end{proposition}
	
	\section*{Диференциране на композиция}
	Нека $x_0 \in U \subset \mathbb{R}^n - $ отворено, 
	\quad $y_0 = f(x_0)\in V\subset \mathbb{R}^m - $ отворено.\\
	$f:U\rightarrow V \quad  g:V\rightarrow \mathbb{R}^k$
	\begin{proposition}
		Нека $g \circ f : U \rightarrow \mathbb{R}^k, \spc$
		$f, g$ - диференцируеми в $x_0$. \\
		Тогава $g \circ f$ е диференцируема в $x_0$ и $d(g\circ f)(x_0) = dg(f(x_0))\circ df(x_0)$. 
	\end{proposition}
	%BEGIN PROOF
	\underline{Доказателство}:
	\textit{To be added...}\\
	%END PROOF
	
	\section*{Инвариантност на формата на диференциала}
\end{document}
