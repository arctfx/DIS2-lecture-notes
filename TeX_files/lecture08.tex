 \documentclass[12pt]{article}
\usepackage[a4paper, margin=2.5cm]{geometry}
\usepackage[T2A,T1]{fontenc}
\usepackage[utf8]{inputenc}
\usepackage[russian,english]{babel}

\usepackage{amsmath}
\usepackage[customcolors]{hf-tikz}
\usepackage{tikz}
\usepackage{tcolorbox}
\usepackage{amsfonts}
\usepackage{mathtools}
\usepackage{graphicx}
\usepackage{epstopdf}
\usepackage{amssymb}
\usepackage{cancel}
\usepackage{hf-tikz}
\usepackage{pgfplots}
\usepackage[utf8]{inputenc}
\DeclareUnicodeCharacter{25A9}{\dash}
\usepackage{color}
\definecolor{darkgray}{gray}{0.1}

%TikZ hf-tikz
\hfsetfillcolor{teal!50!black}
\hfsetbordercolor{teal!80}

%halfbox
\newtcbox{\rbbox}[1][]{on line, sharp corners, colframe=white,coltext=white, colback=darkgray, size=small, toprule=0pt, leftrule=0pt, #1}
\newcommand{\halfbox}[1]{\rbbox{#1}\quad}


\everymath{\displaystyle}

%Riemann integrals
\def\upint{\mathchoice%
	{\mkern13mu\overline{\vphantom{\intop}\mkern7mu}\mkern-20mu}%
	{\mkern7mu\overline{\vphantom{\intop}\mkern7mu}\mkern-14mu}%
	{\mkern7mu\overline{\vphantom{\intop}\mkern7mu}\mkern-14mu}%
	{\mkern7mu\overline{\vphantom{\intop}\mkern7mu}\mkern-14mu}%
	\int}
\def\lowint{
	\mkern3mu\underline{\vphantom{\intop}\mkern7mu}\mkern-10mu\int}
\newcommand{\RNum}[1]{\uppercase\expandafter{\romannumeral #1\relax}}

%NEWTHEOREMS
\newtheorem{proposition}{Твърдение}
\newtheorem{definition}{Def.}
\newtheorem{lemma}{Lemma}
\newtheorem{theorem}{Th.}
\newcommand{\suma}[2]{\overset{#2}{\underset{#1}{\sum}}}
\newcommand{\spc}{\text{ }}

%BEGIN DOCUMENT
\begin{document}
	\selectlanguage{russian}
	\color{white}
	\pagecolor{darkgray}
	\title{Записки по ДИС2 - Лекция 8}
	\date{20.04.2023}
	\maketitle
	\begin{center}
		\Large
		\textbf{Редици и редове от функции.}
	\end{center}
	
	%###########################################################
	$f_n : D \rightarrow \mathbb{R}$, $D\subset\mathbb{R}$, $n\in\mathbb{N}$ \\
	$D' := \{x\in D : \exists \lim\limits_{n\to \infty}f_n(x)\}$ - област на (поточкова) сходимост на $\{f_n\}_{n=1}^\infty$ \\
	$f(x) := \lim\limits_{n\to\infty}f_n(x)$, $x\in D'$ \\
	$f: D' \rightarrow \mathbb{R}$ \\
	
	\begin{definition}
		\textbf{Поточкова граница на $\{f_n\}_{n=1}^{\infty}$} \\
		$f, f_n: D \rightarrow \mathbb{R}$, $D\subset \mathbb{R}$, $n\in\mathbb{N}$ \\
		Казваме, че $\{f_n\}_{n=1}^{\infty}$ \textit{равномерно клони} към $f$ в $D$ (и пишем $f_n\underset{n\to\infty}{\rightrightarrows}$), ако \\
		\[\forall\varepsilon >0 \spc\exists n_0 \in \mathbb{N} \spc\forall n \geq n_0 \spc\forall x \in D : |f_n(x) - f(x)|<\varepsilon\]		
		\[
		\tikzmarkin{z}(0.15,-0.2)(-0.15,0.4)
		n_0(\varepsilon)
		\tikzmarkend{z}
		^{\scalebox{1.25}{$\curvearrowleft$}} n_0 \text{ зависи само от }\varepsilon
		\]\\
	
		$f$ е поточкова граница на $\{f_n\}_{n=1}^{\infty}$ в $D$:
		\[\forall x \in D \underbracket[0.187ex]{\forall \varepsilon >0 \spc\exists n_0 \in \mathbb{N} \spc\forall n \geq n_0 : |f_n(x)-f(x)|<\varepsilon}_{\text{дефиниция за граница на числови редове}}\]
		\[
		\tikzmarkin{z2}(0.15,-0.2)(-0.15,0.4)
		n_0(\varepsilon , x) 
		\tikzmarkend{z2}
		^{\scalebox{1.25}{$\curvearrowleft$}} n_0 \text{ зависи от }\varepsilon\text{ и от }x
		\]
	\end{definition}
	
	$f_n(x)=\frac{1}{1+x^{2n}}$\\
		\begin{tikzpicture}
			\begin{axis}[x=1.5cm,y=4cm]		
				\addplot[	cyan,
							domain=-4:4,
							samples=400,
						] 	{1/(1+x^(24))};
				\addplot[	cyan,
							domain=-4:4,
							samples=200,
						]	{1/(1+x^(10))};
				\addplot[	cyan,
							domain=-4:4,
							samples=200,
						]	{1/(1+x^(4))};
				\addplot[white, dashed] coordinates {(-1,0) (-1,1)};
				\addplot[white, dashed] coordinates {(1,0) (1,1)};
				\addplot[white] coordinates {(-4,0) (-1,0)};
				\addplot[white, mark=o] coordinates {(-1,0)};
				\addplot[white, mark=*] coordinates {(-1,0.5)};
				\addplot[white, mark=o] coordinates {(-1,1) (1,1)};
				\addplot[white, mark=o] coordinates {(1,0)};
				\addplot[white, mark=*] coordinates {(1,0.5)};
				\addplot[white] coordinates {(1,0) (4,0)};
			\end{axis}
		\end{tikzpicture}
	
	\halfbox{Наблюдение} $\|f - f_n\|_\infty := \underset{x\in D}{sup}\spc|f_n(x) - f(x)|$ \\
	Твърдим, че $f_n\underset{n\to\infty}{\rightrightarrows}f$ в $D$ точно тогава, когато $\|f - f_n\|_\infty \underset{n\to\infty}{\longrightarrow}0$, т.е. 
	\[\forall\varepsilon>0\spc\exists n_0 \in \mathbb{N}\spc\forall n \geq n_0 : \underset{x\in D}{sup}\spc|f_n(x) - f(x)| < \varepsilon\]
	\quad\quad\boxed{\varepsilon>0} \\
	\begin{equation*}
		\begin{rcases}
			\varepsilon>2 \\
			f_n\underset{n\to\infty}{\rightrightarrows}f \text{ в }D  \\
		\end{rcases}
		\exists n_0 \in \mathbb{N}\spc\forall n \geq n_0 \underbrace{\forall x \in D : |f_n(x) - f(x)| < \frac{\varepsilon}{2}}_{\underset{x\in D}{sup}|f_n(x) - f(x)| \leq \frac{\varepsilon}{2}<\varepsilon}
	\end{equation*}

	\begin{flalign*}
		\halfbox{Пример} &f_n(x) = \frac{sin(x)}{nx}\underset{n\to\infty}{\longrightarrow}0&\\
		\forall x \in \mathbb{R} \quad &f_n(x) \equiv 0&\\
		&\underset{x\in\mathbb{R}}{sup}|f_n(x) - f(x)| = \underset{x\in\mathbb{R}}{sup}|\textstyle{\frac{sin\spc nx}{n}}| \leq \frac{1}{n}\underset{n\to\infty}{\longrightarrow}0
	\end{flalign*}\\
	\halfbox{Пример} $1+x+x^2+...+x^n+...\spc ,\quad x\neq 1$ \\
	$s_n(x) = 1 + x + x^2 + ... + x^n =
	\tikzmarkin[set fill color=gray!30!black,
	set border color=gray!80!black]{z3}(0.05,-0.4)(-0.05,0.7)
	\frac{1-x^{n+1}}{1-x}
	\tikzmarkend{z3}
	\underset{n\to\infty}{\longrightarrow}\frac{1}{1-x}$ \\ \textcolor{gray}{Тук ползвахме формулата за геометрична прогресия: $\spc\tikzmarkin[set fill color=gray!30!black,
		set border color=gray!80!black]{z4}(0.1,-0.4)(-0.1,0.7)
		s_n=\frac{b_1(1-q^{n})}{1-q}
		\tikzmarkend{z4}$} \\
	$|x| < 1 \Rightarrow D' = (-1, 1)$
	
	\boxed{\text{Фиксираме }n} $\underset{x\in(-1,1)}{sup}|\textstyle{s_n(x) - \frac{1}{1-x}}| = \underset{x\in(-1,1)}{sup}\frac{1-x^{n+1}}{1-x}-\frac{1}{1-x} = \underset{x\in(-1,1)}{sup}\frac{|x|^{n+1}}{1-x} = +\infty$ \\
	\\
	\halfbox{Пример} $f_n(x) = x^n(1-x),\quad x\in[0,1]$ \\
	
	\boxed{\text{Фиксираме }x}
	$x^n(1-x)\underset{n\to\infty}{\longrightarrow}0$\\
	
	\boxed{\text{Фиксираме }n}
	$\underset{x\in[0,1]}{sup}|x^n(1-x)-0|=\underset{x\in[0,1]}{sup}(x^n(1-x))$\\
	$\textstyle{(x^n - x^{n+1})' = nx^{n-1} - (n+1)x^n = x^n(\frac{n}{x}-n-1)=x^n(\frac{n-xn-x}{x})}$
	\begin{flushleft}
		$n-xn-x=0 \Rightarrow x=\tikzmarkin{z5}(0.15,-0.4)(-0.1,0.55)
	\frac{n}{n+1}
	\tikzmarkend{z5}^{\scalebox{1.25}{$\curvearrowleft$}} \text{критична точка}$
	\end{flushleft}
	
	\begin{flushleft}
		$\left(\frac{n}{n+1}\right)^n\left(\frac{1}{n+1}\right) = \left(\frac{1}{\frac{1}{n}+1}\right)^n\left(\frac{1}{n+1}\right) \underset{n\to\infty}{\longrightarrow}0$
	\end{flushleft}
	 	
	\halfbox{Задача} Да се намери областтна на сходимост на $nx^n(1-x)$ в $[0,1]$\\
	
	\begin{theorem}
		Равномерната граница на редица от непрекъснати функции е непрекъсната функция.
	\end{theorem}
	$f, f_n : D \rightarrow \mathbb{R}, \quad D\subset\mathbb{R}, \quad n\in\mathbb{N},\quad f_n \underset{n\to\infty}{\rightrightarrows}f$ в $D$, $x_0 \in D$\\
	$f_n$ - непрекъсната в $x_0 \quad \forall n \in \mathbb{N}$.\quad Тогава $f$ е непрекъсната в $x_0$.\\
	
	%BEGIN PROOF
	\textbf{\underline{Доказателство:}}
	$|f(x) - f(x_0)| \leq |f(x)-f_n(x)| + |f_n(x)-f_n(x_0)| + |f_n(x_0) - f(x_0)|$\\
	Нека $\varepsilon$ - произволно. $f_n \rightrightarrows f$ в $D \Rightarrow \boxed{\exists n_0 \in \mathbb{N}}\spc \forall n \geq n_0 \forall x \in D : |f_n(x) - f(x)| < \frac{\varepsilon}{3}$\\
	$f_{n_0}\text{ e непрекъсната в }x_0 \Rightarrow \boxed{\exists\delta>0}\spc\forall x\in D, |x - x_0| < \delta : |f_{n_0}(x) - f_{n_0}(x_0)| < \frac{\varepsilon}{3}$ \\
	Значи за всяко $x\in D, \spc|x - x_0|<\delta$ е в сила $|f(x) - f(x_0)| \leq |f(x) - f_{n_0}(x)| + |f_{n_0}(x) - f_{n_{0}}(x_0)| + |f_{n_0}(x_0) - f(x_0)| < \frac{\varepsilon}{3} + \frac{\varepsilon}{3} + \frac{\varepsilon}{3} = \varepsilon.$ \hfill$\blacksquare$\\
	%END PROOF
	
	\begin{theorem}
		\textbf{Необходимо и достатъчно условие на Коши за равномерна сходимост}
	\end{theorem}
	Нека $f: D \rightarrow \mathbb{R},\spc D\subset\mathbb{R},\spc n\in\mathbb{N}$. Твърдим, че $\exists f:D\rightarrow\mathbb{R}$ такава, че $f_n\underset{n\to\infty}{\rightrightarrows}f$ в $D$ точно тогава, когато $\forall \varepsilon > 0 \spc\exists n_0 \in \mathbb{N} \spc\forall n \geq n_0 \spc\forall p \in \mathbb{N} \spc\forall x \in D : |f_n(x) - f_{n + p}(x)| < \varepsilon$. ($n_0$ зависи само от $\varepsilon$)\\
	
	%BEGIN PROOF
	\textbf{\underline{Доказателство:}}
	\textbf{($\Rightarrow$)}
	\begin{equation*}
		\begin{rcases}
			f_n\underset{n\to\infty}{\rightrightarrows}f \text{ в } D \\
			\varepsilon > 0
		\end{rcases}
		\Rightarrow \boxed{\exists n_0 \in \mathbb{N}}\spc\forall n \geq n_0 \spc\forall x \in D : |f_n(x) - f(x)| < \frac{\varepsilon}{2}.
	\end{equation*}
	$n\geq n_0,\spc p\in\mathbb{N}(n+p\geq n_0),\spc x\in D$\\
	$|f_n(x)-f_{n+p}(x)|\leq |f_n(x)-f(x)|+|f(x)-f_{n+p}(x)|<\frac{\varepsilon}{2}+\frac{\varepsilon}{2}=\varepsilon$\\
	\textbf{($\Leftarrow$)} \boxed{x\in D} - фиксираме, $\quad \{f_n(x)\}_{n=1}^{\infty}\subset\mathbb{R}$ е фундаментална $\rightarrow \{f_n(x)\}_{n=1}^{\infty}$ - сходяща за всяко $x\in D,\quad f(x) := \lim\limits_{n\to\infty}f_n(x),\quad f:D\rightarrow \mathbb{R}$\\
	Да проверим, че $f_n\underset{n\to\infty}{\rightrightarrows}f$ в $D$\\
	\boxed{\varepsilon>0} $\Rightarrow \exists n_0\in \mathbb{N} \spc\forall n \geq n_0 \spc\forall p \in \mathbb{N} \spc\forall x \in D : |f_n(x) - f_{n+p}(x)|<\frac{\varepsilon}{2}$ (След граничен преход $p \rightarrow +\infty$) $\underline{\forall n \geq n_0 \spc\forall x \in D}: |f_n(x)-f(x)|\leq \frac{\varepsilon}{2}<\varepsilon.$
	\hfill$\blacksquare$\\
	%END PROOF
	$\spc$\\
	$\spc$\\
	\textit{За следната теорема \textbf{няма да даваме доказателство}:}
	\begin{theorem}
		\textbf{НДУ за равномерна сходимонст на ред на Коши}\\
		$\sum_{n=1}^{\infty}f_n(x),\spc f_n:D \rightarrow \mathbb{R},\spc D\subset \mathbb{R},\spc n\in\mathbb{N}\spc$ Тогава редът $\sum_{n=1}^{\infty}f_n(x)$ е равномерно сходящ в $D \Leftrightarrow \forall \varepsilon > 0 \spc\exists n_0 \in \mathbb{N} \spc\forall n\geq n_0 \spc\forall p \in \mathbb{N} \spc\forall x \in D: \big|\textstyle{\sum_{i=n+1}^{n+p}f_i(x)}\big|=\big|S_{n+p}(x) - S_n(x)\big|<\varepsilon$.
	\end{theorem}
	\begin{theorem}
		\textbf{Критерий на Вайерщрас}\\
		$\sum_{n=1}^{\infty}f_n(x),\spc f_n:D \rightarrow \mathbb{R},\spc D\subset \mathbb{R},\spc n\in\mathbb{N}\quad$ $|f_n(x)|\leq a_n \spc\forall x \in D,\spc \sum_{n=1}^{\infty}a_n$ е сходящ.\\ Тогава $\sum_{n=1}^{\infty}a_n$ е равномерно (и абсолютно) сходящ в $D$.
	\end{theorem}
	%BEGIN PROOF
	\textbf{\underline{Доказателство:}}
	\boxed{\varepsilon>0}\\
	$\sum_{n=1}^{\infty}a_n$ е сходящ $\Rightarrow \exists n_0 \in \mathbb{N} \spc\forall n \geq n_0 \spc\forall p \in \mathbb{N} : \big|\textstyle{\sum_{i=n+1}^{n+p}a_i}\big|<\varepsilon$\\
	Тогава за $n\geq n_0$ - произволно, $p \in \mathbb{N}$, $x\in D$ - произволно, имаме:\\
	$\big|\textstyle{\sum_{i=n+1}^{n+p}f_i(x)}\big|\leq \sum_{i=n+1}^{n+p}\big|f_i(x)\big|\leq \sum_{i=n+1}^{n+p}a_i < \varepsilon.$
	\hfill$\blacksquare$\\
	%END PROOF
	\halfbox{Пример} $\sum_{n=1}^{\infty}\frac{x^n}{n^2}$
	$\quad x\in [-1,1]\longrightarrow \big|\frac{x^n}{n^2}\big|\leq \frac{1}{n^2},\spc \sum_{n=1}^{\infty}\frac{1}{n^2}$ - сходящ.
	\begin{theorem}
		$\Delta - \text{ ограничен интервал},\spc f_n : \Delta \rightarrow \mathbb{R}, n\in\mathbb{N}$ - диференцируеми. $\spc \{f_n(x)\}_{n=1}^{\infty}$ е равномерно сходящ в $\Delta$. Съществува $x_0 \in \Delta, \{f_n(x_0)\}_{n=1}^{\infty}$ е сходяща.\\
		Тогава редицата $\{f_n(x)\}_{n=1}^{\infty}$ е равномерно сходяща в $\Delta$, $\lim\limits_{n\to\infty}f_n$ е диференцируема в $\Delta$ и $\big(\lim\limits_{n\to\infty}f_n(x)\big)' = \lim\limits_{n\to\infty}f'_n(x)\quad \forall x \in \Delta$.
	\end{theorem}
	
\end{document}
